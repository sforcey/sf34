%&biglatex
\documentclass[10pt]{article}
\usepackage{latexsym,amsmath}
\usepackage{graphics}
\usepackage{graphicx}
\usepackage{rotating}
\usepackage[all]{xy}
\xyoption{v2}
\xyoption{knot}
\usepackage{epsfig}
\oddsidemargin  0pt     % 
\evensidemargin 0pt     % 
\marginparwidth 1in     % 
\marginparsep 0pt       % 
\topmargin 0pt          % 
\headheight 0pt         % 
\headsep 0pt            % 
\topskip 0pt            % 
\footskip 0.5in         % 
\textwidth 6.5in        % 

\textheight 9in
\newtheorem{theorem}{Theorem}[section]
\newtheorem{proposition}[theorem]{Proposition}
\newtheorem{corollary}[theorem]{Corollary}
\newtheorem{lemma}[theorem]{Lemma}
\newtheorem{conjecture}[theorem]{Conjecture}
\newtheorem{hypothesis}{IH}

\newenvironment{definition}
{\bigskip\refstepcounter{theorem}\noindent{\bf Definition \thetheorem.}}
{\bigskip}

\newenvironment{remark}
{\bigskip\refstepcounter{theorem}\noindent{\bf Remark \thetheorem.}}
{\bigskip}

\newenvironment{exercise}
{\bigskip\refstepcounter{theorem}\noindent{\bf Exercise \thetheorem.}}
{\bigskip}

\newenvironment{example}
{\bigskip\refstepcounter{theorem}\noindent{\bf Example \thetheorem.}}
{\bigskip}

\newenvironment{proof}
{\noindent{\bf Proof}}
{\bigskip}

\newcommand{\MySection}[1]
{\noindent{\bf #1}}

\newcommand{\bcal}[1]{\mbox{\boldmath${\cal {#1}}$}}
%\setcounter{totalnumber}{5}

\begin{document}
%\vspace{0.5cm}
 \begin{center}
   {\bf\large Research Statement}
   
  % \vspace{ 1cm}
   
   {\large Stefan Forcey}
   \end{center}
   
\MySection{Overview}

My studies involve the development of extended
topological field theories and the relationship of categorical 
enrichment to iterated loop spaces, as well as applications of these
concepts to other areas of mathematics and physics. The major focuses of my research thus far are 
interrelated as in the following diagram, where the directions of arrows
indicate
flow of information and the numerals correspond to sections in the following text.
\begin{small}
$$
\xymatrix@d@C=-10pt@R=-40pt{
 &*\txt{homotopy theory:\\iterated loop spaces\\(and their operads)}
 \ar@{<->}[ddl]
 \ar@{<->}[ddr]
 \\
 &\text{(III)}\text{ }\text{ }\text{ }\text{ }\text{ }\text{ }
 \\
 *\txt{enriched\\\text{ }categories\text{ }\\\text{ }}
 \ar@{<->}[rr]
 \ar[ddr]
 &&*\txt{(weak)\\$n$--categories}
 \ar[ddl]
 \\
 &\text{ }\text{ }\text{ }\text{ }\text{ }\text{ }\text{(II)}
 \\
 &*\txt{(I)\\\text{ }extended\text{ }\\TQFT}
 \ar[ddr]
 \ar[ddl]
 \\
 &\text{(IV)}\text{ }\text{ }\text{ }\text{ }\text{ }\text{ }
 \\
 *\txt{knot\\\text{ }concordance\text{ }\\\text{ }}
 \ar[rr]
 &&*\txt{\text{ }loop quantum gravity\text{ }\\and string theory}
 }
$$
\end{small}
\MySection{I. Extended Topological Quantum Field Theories}

Naively a TQFT is a topological invariant of spaces that is functorial
with respect to cobordisms.
This idea is motivated by physics where we are used to picturing a pair of
spaces forming the boundary of a space-time
cobordism. Let $d+1$
be the dimension of the spacetime.
Technically a $d+1$ dimensional TQFT is a multiplicative functor between bordism
categories. The domain is usually the topological spaces
to be studied, with multiplication disjoint union, and the range is
usually a simpler monoidal category, such as 
modules over a given ring, with multiplication the tensor product. An $n$--category
has  arrows (1--cells) between objects (0--cells), 2--cells between 1--cells, and so on. 
A multidimensional TQFT that has an $n$--category as its range is 
known as an extended topological field theory. The picture that we hope to realize 
in some fashion is as follows:

\begin{figure}[h]
 %\begin{center}
 \centering
 \scalebox{.65}{\includegraphics{tqftfin.eps}}
 %\end{center}
 \end{figure}
   We studied a new variation on the definition of field theory that retains 
 the gluing properties of a TQFT but defines
the invariants as limits of functors rather than as functors themselves.
Sheaf cohomology is defined as the limit of a presheaf. The latter is a functor from
a category of subsets of a topological space (maps are inclusions) to a category
of modules.
The extended nature of the new field theories is achieved by letting the image of
the $d+1$ dim. cobordism be its sheaf cohomology. The necessary
presheaf is constructed by mapping a time interval subset
of the bordism with a constant cross section to the image of the underlying TQFT on that cross section.

Personally I focused on defining the
new theory on $2+1$ dim. cobordisms. The cross sections
of such a $3$--manifold with corners are surfaces with boundary. In order to define an invariant
of these as a limit of the hom--functor we need $2$--dimensional hom--sets, or more generally
multipliable hom--objects. Indicated as the
necessary tools were the concepts of enrichment and $n$--categories.

\MySection{II. Iterated Enrichment and Higher Dimensional Algebra}

The definition of enriched category generalizes the usual definition of
category by replacing the hom--sets of morphisms between each two objects by 
hom--objects in some monoidal category ${\cal V}$. The collection of enriched categories over a
given monoidal category, with
enriched functors and natural transformations,  is the $2$--category known
as ${\cal V}$--Cat. ${\cal V}$--Cat is known to inherit
structure from ${\cal V}$: if ${\cal V}$ is symmetric then so is ${\cal V}$--Cat, if ${\cal V}$ is merely braided
then
${\cal V}$--Cat is merely monoidal. These facts are important to the construction
of extended TQFT's since accurately modeling various dimensions requires the 
range category to have various levels of structure, especially if it is desired that 
the field theory reflect the embedding of the cobordism as well as the topology.
 
My contributions to enrichment theory are related to this inheritance of properties.
I defined enrichment over a very general type of monoidal category with
extra structure; a $k$--fold monoidal, or iterated monoidal category. I then proved that
for ${\cal V}$  $k$-fold monoidal ${\cal V}$--Cat is a $(k-1)$--fold monoidal $2$--category. With
some additional constraints on the structure of ${\cal V}$ this should provide a source of
braided monoidal $2$--categories and bicategories.
Next I recursively defined higher dimensional enrichment. ${\cal V}$--$n$--Cat is the collection
of categories enriched over ${\cal V}$--$(n-1)$--Cat. I defined the enriched higher morphisms
for these objects and showed that ${\cal V}$--$2$--Cat is a strict $3$--category. In the future
I hope to complete an inductive proof that for ${\cal V}$ $k$--fold monoidal we have that
${\cal V}$--$n$--Cat is a $(k-n)$--fold monoidal strict $(n+1)$--category.
 The algebraic extension of these results will be to weaken the concept of higher dimensional
 enrichment by replacing
 equalities with isomorphic higher morphisms. Since a strict $n$--category is just a category enriched
over $(n-1)$--Cat, this weakening may also have a direct application
    to the definition of weak $n$--categories, something that is still not
    satisfactorily understood. I hope to be able to contribute to their 
    study next June as I have been invited to the workshop on ``$n$--categories: Foundations
    and Applications'' organized by John Baez and Peter May at the IMA.
The topological extension of my results is to understand 
their implications for
homotopy theory. 

 \MySection{III. Enrichment and Delooping} 
 
 The effort to link category theory and geometry, especially categorical and homotopical coherence,
     has recently been aided by the work of Balteanu,
    Fiedorowicz, Schw${\rm \ddot a}$nzl, and Vogt
    who show a direct correspondence between $k$--fold monoidal
    categories and  $k$--fold loop spaces through the
    categorical nerve.\cite{Balt} The latter is a simplicial complex constructed from a category by
    associating $n$ composable morphisms with $n$-simplices as follows:
     
     \begin{figure}[h]
     %\begin{center}
     \centering
     \scalebox{.6}{\includegraphics{nervefin.eps}}
     %\end{center}
 \end{figure}
     As I pursued my plan to relate the enrichment functor to topology, I
    noticed that the concept of higher dimensional
    enrichment would be important in its relationship to double, triple and
    further iterations of delooping. This is evident immediately from the proof that
    under enrichment categorical dimension increases while monoidal-ness decreases.
    In $\Omega X$ points (objects) are paths (1--cells) in $X$, and loop spaces are always provided
    with a multiplication of points by concatenation of loops. 
    In the future I plan to characterize
    the nerves of ${\cal V}$--$n$--Cat. The weaker version of enrichment also deserves study.
    The nerves of bicategories
    have been defined by Duskin in \cite{Dusk} and may provide the link to loop spaces in this context.
    Other categories based on a monoidal category
    ${\cal V}$ are ${\cal V}$--Act, the bicategory ${\cal V}$--Mod and internal categories in ${\cal V}.$
    All these have potential of being generalized in my scheme and thus
    have promise of being important to homotopy theory. 
 
 \MySection{IV Applications} 
 
 Next I return to the question of field theories that reflect embedding. Baez and Langford have described
 the use of braided bicategories to model braided surfaces \cite{BaezLang}. I would like to apply a braided ${\cal V}$--Mod to 
 their construction and look for useful invariants of knot cobordisms. The knot concordance
 group has elements equivalence classes of knots related in that two equivalent knots
 cobound a cylinder that is smoothly embedded in Euclidean 4--space. The group operation
 is connected sum. The knots that are concordant to the identity unknot are the slice knots--they occur as 
 cross sections of a sphere smoothly embedded in 4--space. 
 Here are examples for the figure eight and the trefoil of a
 drawing algorithm that I constructed to show by picture how the connected sum of a 
 knot and its orientation reversed mirror image forms a special slice knot called a ribbon
 knot. The inverse knot is drawn behind the original by placing the reversed crossings each beside 
 the corresponding original crossing and then connecting arcs in the background. 
  
  \begin{figure}[h]
      %\begin{center}
      \centering
      \scalebox{.4}{\includegraphics{ribbonfin.eps}}
      %\end{center}
 \end{figure}
   It is unknown whether all slice knots are in fact ribbon knots. Amphicheiral knots represent 
 elements of order two in the group,
 but it is unknown whether there are any elements of order two without an amphicheiral representative.
 The trefoil represents an element of infinite order, but it is unknown whether there are elements of order $n\ge 3.$
 It seems worth the effort to attack some of these questions with the new invariants available through 
 quantum field theories. The question of order, for instance, might be solved if it was possible to
 construct a field theory for which the image of a potential cobordism between the unknot and
 a triple connected sum could be shown not to exist. Another way to extend the ideas of TQFT is to 
 define field theories that are invariants of gropes and other CW complexes. The work of Conant and Teichner
 in \cite{Teich} desribes a filtration of the knot concordance group by grope cobordism,
 and this might be the correct place to try to solve problems using such field
 theories as tools.
  
  The connection between TQFT and quantum gravity is
  that field theories often 
  occur as topological state sums, which are invariants calculated from but independent of
  triangulations on the cobordism
  manifold. These state sums are the starting point for loop quantum gravity in the work of several 
  theorists, including Crane, Barrett \cite{BC}, and Smolin \cite{Smolin}. There is
  also an intriguing possibility of applying the sheaf theoretic version of field theory
  to string theory. It has been
  suggested that the transport of strings and higher dimensional branes be described
  in terms of 
  $n$--gerbes. An $0$--gerbe is a sheaf, and a $1$--gerbe assigns 
  groupoids to the open sets. I would like to define gerbe theoretic TQFT and investigate its
  relationship to the $K$--theory of gerbes.
  
 Kauffman notes a connection between the knot concordance group and string theory in \cite{Kauff}. 
  The connected sum of a knot and its inverse is cobordant to the unknot in a way 
  that suggests a particle-antiparticle collision. The cobordism is seen as a movie of cross-sectional
  stills. 
  We include the connected sum as an initial saddle and neglect to cap off
  one of the unknots that results. 
  
  \begin{figure}[h]
        %\begin{center}
        \centering
        \scalebox{.3}{\includegraphics{coolfin.eps}}
        %\end{center}
 \end{figure}
  I would like to define extended topological field theories on this sort of picture, as well as 
 understand any physical significance its higher dimensional analogues may reflect. There may be some
 illumination on the subject in the recent work of Conant and Teichner on grope cobordism and
 Feynman diagrams \cite{CT}.
 
 \begin{thebibliography}{99}
 \bibitem{BaezLang}{J. Baez, L. Langford, Higher-Dimensional Algebra IV:2-Tangles, to appear in Adv. Math. }
 \bibitem{Balt}{C. Balteanu, Z. Fiedorowicz, R. Schw${\rm \ddot a}$nzl, R. Vogt, 
     Iterated Monoidal Categories,
    Adv. Math. 176 (2003), 277-349.}
    \bibitem{BC}{J.W. Barrett, L. Crane, Relativistic Spin Networks and Quantum Gravity, /gr-qc/9709028 }
    \bibitem{Teich}{J. Conant, P. Teichner, Grope Cobordism of Classical Knots,
    math.GT/0012118}
    \bibitem{CT}{J. Conant, P. Teichner, Grope Cobordism and Feynman Diagrams, math.GT/0209075 }
    \bibitem{Dusk}{J.W.Duskin, Simplicial Matrices and the Nerves of Weak n--Categories I: Nerves of Bicategories,
  Theory and Applications of Categories, 9, No. 10 (2002), 198-308.}
  %\bibitem{FW}{D. Freed, E. Witten, Anomalies in String Theory with $D$--branes, Asian J. Math, 3 (1999), 819-851.}
  \bibitem{Kauff}{L. Kauffman, Knots and Physics 3rd ed., World Scientific Press, 2001}
   \bibitem{Smolin}{L. Smolin, Linking Topological Quantum Field Theory and Nonperturbative Quantum Gravity,
   J. Math. Phys., 36 (1995), 6417-6455.}
 \end{thebibliography}
 \end{document}