%&biglatex
\documentclass[10pt]{article}
\usepackage{latexsym,amsmath}
\usepackage{graphics}
\usepackage{graphicx}
\usepackage{rotating}
\usepackage[all]{xy}
\xyoption{v2}
\xyoption{knot}
\usepackage{epsfig}
\oddsidemargin  0pt     % 
\evensidemargin 0pt     % 
\marginparwidth 1in     % 
\marginparsep 0pt       % 
\topmargin 0pt          % 
\headheight 0pt         % 
\headsep 0pt            % 
\topskip 0pt            % 
\footskip 0.5in         % 
\textwidth 6.5in        % 

\textheight 9in
\newtheorem{theorem}{Theorem}[section]
\newtheorem{proposition}[theorem]{Proposition}
\newtheorem{corollary}[theorem]{Corollary}
\newtheorem{lemma}[theorem]{Lemma}
\newtheorem{conjecture}[theorem]{Conjecture}
\newtheorem{hypothesis}{IH}

\newenvironment{definition}
{\bigskip\refstepcounter{theorem}\noindent{\bf Definition \thetheorem.}}
{\bigskip}

\newenvironment{remark}
{\bigskip\refstepcounter{theorem}\noindent{\bf Remark \thetheorem.}}
{\bigskip}

\newenvironment{exercise}
{\bigskip\refstepcounter{theorem}\noindent{\bf Exercise \thetheorem.}}
{\bigskip}

\newenvironment{example}
{\bigskip\refstepcounter{theorem}\noindent{\bf Example \thetheorem.}}
{\bigskip}

\newenvironment{proof}
{\noindent{\bf Proof}}
{\bigskip}

\newcommand{\MySection}[1]
{\noindent{\bf #1}}

\newcommand{\bcal}[1]{\mbox{\boldmath${\cal {#1}}$}}
%\setcounter{totalnumber}{5}

\begin{document}
%\vspace{0.5cm}
 \begin{center}
   {\bf\large Curriculum Vitae}
   
  % \vspace{ 1cm}
   
   {\large Stefan Forcey}
   \end{center}
$$
\xymatrix@C=150{
*\txt{{\bf Department of Mathematics\text{ }\text{ }\text{ }\text{ }\text{ }\text{ }\text{ }\text{ }}\\
460 McBryde Hall\text{ }\text{ }\text{ }\text{ }\text{ }\text{ }\text{ }\text{ }\text{ }\text{ }\text{ }\text{ }\text{ }\text{ }\text{ }\text{ }\text{ }\text{ }\text{ }\text{ }\text{ }\text{ }\text{ }\text{ }\text{ }\text{ }\\
Blacksburg, VA 24061-0123\text{ }\text{ }\text{ }\text{ }\text{ }\text{ }\text{ }\text{ }\text{ }\text{ }\text{ }\text{ }\text{ }\text{ }\text{ }\text{ }\\
Phone: (540) 231-4082\text{ }\text{ }\text{ }\text{ }\text{ }\text{ }\text{ }\text{ }\text{ }\text{ }\text{ }\text{ }\text{ }\text{ }\text{ }\text{ }\text{ }\text{ }\text{ }\text{ }\text{ }\text{ }\\
URL: www.math.vt.edu/people/sforcey}
&
*\txt{{\bf Home\text{ }\text{ }\text{ }\text{ }\text{ }\text{ }\text{ }\text{ }\text{ }\text{ }\text{ }\text{ }\text{ }\text{ }\text{ }\text{ }\text{ }\text{ }\text{ }\text{ }\text{ }\text{ }\text{ }\text{ }\text{ }\text{ }\text{ }\text{ }\text{ }\text{ }}\\
311 D Reynolds St.\text{ }\text{ }\text{ }\text{ }\text{ }\text{ }\text{ }\text{ }\text{ }\text{ }\text{ }\text{ }\text{ }\text{ }\text{ }\text{ }\text{ }\\
Blacksburg, VA 24060\text{ }\text{ }\text{ }\text{ }\text{ }\text{ }\text{ }\text{ }\text{ }\text{ }\text{ }\text{ }\text{ }\text{ }\text{ }\\
Phone: (540) 961-2471\text{ }\text{ }\text{ }\text{ }\text{ }\text{ }\text{ }\text{ }\text{ }\text{ }\text{ }\text{ }\text{ }\text{ }\\
Email: sforcey@math.vt.edu\text{ }\text{ }\text{ }\text{ }\text{ }\text{ }\text{ }}}\\
$$
\newline

\noindent{\bf Research Interests}

\begin{tabular}{ll}
& Algebraic Topology, Topological Quantum Field Theory, braided categories,\\
&enriched categories, weak $n$--categories, knot concordance, braid groups.
\end{tabular}
\newline

\noindent{\bf Ph.D. Dissertation}

\begin{tabular}{ll}
&Virginia Tech: {\itshape Topological Implications of Enrichment over Iterated Monoidal Categories}\\
&Advisor: Frank Quinn
\end{tabular}
\newline

\noindent{\bf Education}

\begin{tabular}{ll}
$\bullet$&(expected) May 2004 Ph.D. in Mathematics,\\
&from the Mathematical Physics doctoral program at Virginia Tech.\\
$\bullet$&May 2002 M.S. in Mathematics from Virginia Tech.\\
$\bullet$&May 1997 B.S. in Mathematics with Computer Science concentration\\
&from Liberty University, Lynchburg, VA.
\end{tabular}
\newline

\noindent{\bf Publications}

\begin{tabular}{ll}
$\bullet$& {\itshape Enrichment as Categorical Delooping I. Enrichment over Iterated Monoidal Categories}\\
& submitted to Algebraic and Geometric Topology, preprint available math.CT/0304026\\
$\bullet$& {\itshape Higher Dimensional Enrichment}\\
&submitted to Theory and Applications of Categories, preprint available math.CT/0306086\\
$\bullet$& {\itshape Axiomatic Topological Quantum Field Theory} (with F. Quinn, J. Siehler)\\
&to appear in Encyclopedia of Mathematical Physics.\\
\end{tabular}
\newline

\noindent{\bf Conference Talks}

\begin{tabular}{ll}
$\bullet$&{\itshape Enrichment and Delooping of Categories with Loop Space Nerves}\\
&Workshop
on Categorification and Higher-Order Geometry\\
&Instituto Superior T\'{e}cnico, Lisbon, Portugal, July 2003.\\
$\bullet$&{\itshape Braids and Enrichment }\\
&Lehigh University Geometry and Topology Conference,
Bethlehem, PA., June 2003.\\

\end{tabular}
\newline

\noindent{\bf Seminar Talks}

\begin{tabular}{ll}
$\bullet$&{\itshape Twisted Ribbons and Categorical Consequences of the Yang-Baxter Equation}\\
&one lecture for Virginia Tech graduate research seminar, Sep. 2003.\\
$\bullet$&{\itshape Introduction to Topological Quantum Field Theory}\\
&one lecture for VT graduate research seminar, March 2002.\\
$\bullet$&{\itshape Train Tracks and Projective Laminations}\\
&two lectures for VT graduate research seminar, March 2002.\\
$\bullet$&{\itshape Introduction to the Knot Concordance Group}\\
& one lecture for VT topology topics course, April 2000.\\
$\bullet$&{\itshape Basics of Eigenvalue Perturbation Theory}\\
& one lecture for VT functional analysis course, April 2000.\\
\end{tabular}
\newline

\noindent{\bf Conference Participation}

\begin{tabular}{ll}
$\bullet$&(invited) $n$--categories: Foundations
    and Applications\\
    &Institute for Mathematics and its Applications, June 2004. \\
$\bullet$&Young Researchers Symposium
on Mathematical Physics\\
&Instituto Superior T\'{e}cnico, Lisbon, Portugal, July 2003.\\  
$\bullet$&Workshop
  on Categorification and Higher-Order Geometry\\
&Instituto Superior T\'{e}cnico, Lisbon, Portugal, July 2003.\\
$\bullet$&Lehigh University Geometry and Topology Conference, June 2003.\\
\end{tabular}
\newline

\noindent{\bf Conference Participation (Cont.)}

\begin{tabular}{ll}
$\bullet$&University of Arkansas Spring Lecture Series
in the Mathematical Sciences:\\
&The Andrews-Curtis and the Poincare Conjectures, April 2003.\\
$\bullet$&AMS Special Session on Quantum Topology Columbus, OH., Sep. 2001.\\
$\bullet$&Georgia International Topology Conference, UGA, May 2001.\\
$\bullet$&Cornell Topology Festival, May 2001.\\
$\bullet$&(limited) Mid-Atlantic Algebra Conference, Virginia Tech, March 2001. \\
\end{tabular}
\newline

\noindent{\bf Academic Appointments}

\begin{tabular}{ll}
$\bullet$&Graduate Research Assistant, Virginia Tech (supported by NSF),\\
& Summer 2001 and 2002, Summer/Spring/Fall 2003, Spring 2004.\\
$\bullet$&Graduate Teaching Assistant, Virginia Tech \\
&Fall 1997, Spring 1998, Spring/Summer/Fall 1999, Spring/Fall 2000-2002.\\
$\bullet$& Instructor, Virginia Tech Calculus Readiness Week, July 2000.\\
\end{tabular}
\newline

\noindent{\bf Nonacademic Employment}

\begin{tabular}{ll}
$\bullet$&Framatome Technologies, Lynchburg VA. Jan. 1996 - Aug. 1997, Summer and Fall 1998.\\
&\text{ }\text{ }Job description included comparisons of and extrapolation from data measured\\
&in the reactor cores of nuclear power plants as well as computer modeling\\
&of reactors for purposes of high level waste research. 
\end{tabular}
\newline

\noindent{\bf Awards and Memberships}

\begin{tabular}{ll}
$\bullet$&AMS member, 2000-present.\\
$\bullet$&Kappa Mu Epsilon  Mathematics Honor Society member 1994-1997, chapter president 1997.\\
$\bullet$&Freshman Computer Science Award, Liberty University, 1994.\\
$\bullet$&National Merit Finalist, 1993.\\
\end{tabular}
\newline

\noindent{\bf Teaching Experience}

\begin{tabular}{lll}
$\bullet$&Courses taught at Virginia Tech (with full teaching responsibility)\\
&-\text{ }\text{ }Math 1016, Elementary Calculus with Trigonometry I, Calculus including limits, \\
&\text{ }\text{ }\text{ }derivatives, applications of derivatives, trigonometric functions. Computer component.\\
&-\text{ }\text{ }Math 1205,  Calculus, Differential calculus for science and engineering majors.\\
&-\text{ }\text{ }Math 1224, Vector Geometry, Topics in analytic geometry and conic sections,\\
&\text{ }\text{ }\text{ }and the calculus of vector-valued functions. Computer component.\\
$\bullet$&Calculus Readiness Week Instructor.\\
&I prepared a group of students for the calculus entrance exam at VT.\\
$\bullet$&Tutor for Math 1526, Elementary Calculus with Matrices, Integration with applications,\\
& matrix algebra and solving systems of equations, multivariable calculus.\\
\end{tabular}
\newline

\noindent{\bf Professional Development}

\begin{tabular}{ll}
$\bullet$&GTA Training Workshop, Virginia Tech, 1 Credit Hour, Fall 1999.\\
\end{tabular}
\newline

\noindent{\bf Computer Skills}

\begin{tabular}{ll}
$\bullet$&Mathematica and MatLab, {\LaTeX}, the diagram package {\Xy-pic}. \\
$\bullet$&Programming experience: FORTRAN, BASIC, PASCAL, PROLOG, LISP, C++.\\
$\bullet$&Operating Systems: Windows, Mac OS, UNIX, LINUX. Applications: Microsoft Office.\\
\end{tabular}
\newline

\noindent{\bf References}\\
\begin{tabular}{llllll}
&&&Dr. Frank Quinn&Dr. William Floyd&Dr. Peter Haskell\\
&&&Dept. of Mathematics&Dept. of Mathematics&Dept. of Mathematics\\
&&&460 McBryde Hall&460 McBryde Hall&460 McBryde Hall\\
&&&Blacksburg, VA 24061-0123&Blacksburg, VA 24061-0123&Blacksburg, VA 24061-0123\\
&&&quinn@math.vt.edu&floyd@math.vt.edu&haskell@math.vt.edu\\
\end{tabular}

\newline

\begin{tabular}{llllllllllll}
&&&&&&Dr. Lay Nam Chang&&&&Dr. Eileen Shugart\\
&&&&&&Dept. of Physics&&&&Dept. of Mathematics\\
&&&&&&Robeson Hall&&&&460 McBryde Hall\\
&&&&&&Blacksburg, VA 24061-&&&&Blacksburg, VA 24061-0123\\
&&&&&&laynam@phys.vt.edu&&&&shugart@math.vt.edu\\
\end{tabular}
\newline
 \end{document}