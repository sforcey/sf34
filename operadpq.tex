%&biglatex
% Latex 2e document, tac article style, 47 pp, Xy-pic ver 3.7, MiKTeX version 2.1
\documentclass{tac}
\usepackage{rotating}
\usepackage{graphics}
\usepackage[all]{xy}
\xyoption{v2}
\xyoption{knot}


\author {S. Forcey and J. Siehler}

\thanks{Thanks to {\Xy-pic} for the diagrams. }

\address{Department of Mathematics\\
     Virginia Tech\\
     460 McBryde Hall\\
     Blacksburg, VA 24060 \\
     USA\\
}


\title {Combinatoric n-fold categories and n-fold operads}


\copyrightyear{2003}

\keywords{operads, iterated monoidal categories}
\amsclass{18D10; 18D20}

\eaddress{sforcey@math.vt.edu}


%NOTE:  author macros  BEGIN here
%       (they are all actually used in the article!!)

\newtheorem{theorem}{Theorem}
\newtheorem{definition}{Definition}
\newtheorem{example}{Example}

\newcommand{\MySection}[1]
{\section{ #1}}

\newcommand{\bcal}[1]{\mbox{\boldmath${\cal {#1}}$}}
\mathrmdef{Hom}
\mathrmdef{sort}

% author macros END here

\begin{document}
\SloppyCurves{

\maketitle
\begin{abstract}
 Operads were originally defined as ${\cal V}$-operads, that is, enriched in a symmetric or braided 
 monoidal category ${\cal V}.$
 The symmetry or braiding in ${\cal V}$ is required in order
 to describe the associativity axiom the operads must obey, as well as the associativity that
 must be a property of the action of an operad on any of its algebras. A sequence of categorical types that 
 filter the category of monoidal categories and monoidal functors is given by Balteanu, Fiedorowicz, Schwanzl and Vogt in 
 \cite{Balt}. These subcategories of {\bf MonCat} are called $n$-fold monoidal categories. A $k$--fold monoidal category
 is $n$-fold monoidal for all  $n\le k$, and a symmetric monoidal category is $n$-fold monoidal for all $n$.
 After a review of the role of operads in loop space theory and higher categories 
 we go over definitions of iterated monoidal categories and the beginnings of a large family of simple examples.  
 Then we generalize the definition of operad by defining $n$-fold operads and their algebras in an iterated monoidal category.
 We discuss examples of these that live in the previously described categories. Finally we describe 
 the $(n-2)$-fold monoidal category of $n$-fold ${\cal V}$-operads. 
 
 

 \end{abstract}

  %{intro2}
      \clearpage
                  \tableofcontents 
      
            \MySection{Introduction}
            Operads in a category of topological spaces are the crystallization of several approaches
            to the recognition problem for iterated loop spaces. Beginning with Stasheff's associahedra 
            and Boardman and Vaught's little $n$-cubes, and continuing with more general $A_{\infty}$ and
            $E_{\infty}$ operads described by May and others, that problem was largely solved. 
            \cite{Sta}, \cite{BV1}, \cite{May}
            Loop spaces are characterized by admitting an operad action of the appropriate kind. 
            
             Recently there has also been growing interest in the application of higher dimensional structured 
            categories to the characterization of loop spaces. The program being advanced by many categorical
            homotopy theorists seeks to model the coherence laws governing homotopy types with the coherence
            axioms of stuctured $n$-categories.  By modeling we mean a
            connection that will necessarily be in the form of a functorial 
            equivalence
            between categories of special categories and categories of special spaces. The largest challenges 
            currently are to find the most natural and efficient definition of (weak) $n$-category, and to 
            find the right sort of connecting functor. 
            
            One major recent advance is the discovery of Balteanu, Fiedorowicz, Schwanzl and Vogt in 
            \cite{Balt} that the nerve functor on categories gives 
            a direct connection between iterated monoidal categories and iterated loop spaces.
            Stasheff \cite{Sta} and
	        MacLane \cite{Mac} showed that monoidal categories are precisely analogous to 1-fold
	        loop spaces. There is a similar connection between symmetric monoidal categories and
	        infinite loop spaces. The first step in filling in the gap between 1 and infinity was made in
	        \cite{ZF} where it is shown that the group completion of the nerve of a braided monoidal category
	        is a 2-fold loop space.   
	        In \cite{Balt} the authors finished this process by, in their words, ``pursuing an analogy to the tautology 
	        that an $n$-fold loop space is a loop
	        space in the category of $(n-1)$-fold loop spaces.'' The first
	        thing they focus on is the fact that a braided category is a special case of a carefully
	        defined 2-fold monoidal category. Based on their observation of the  correspondence between
	        loop spaces and monoidal categories, they iteratively define the notion of $n$-fold
	        monoidal category as a monoid in the category of $(n-1)$-fold monoidal categories.
	        In \cite{Balt} a symmetric category
	        is seen as a category that is $n$-fold monoidal for all $n$.
	        The main result in that paper is that the group
    completion of the nerve of an $n$-fold monoidal category is an $n$-fold loop space.
            It is still
            open whether this is a complete characterization, that is, whether every $n$-fold loop space arises as the 
            nerve of an $n$-fold category. 
            
            The connection between the $n$-fold monoidal categories of Fiedorowicz and the theory of higher categories
            is through Baez's periodic table.\cite{Baez1} Here Baez organizes the $k$-tuply monoidal $n$-categories, by which
            terminology he refers to
            $(n+k)$-categories  that are trivial
            below dimension $k.$ The triviality of lower cells allows the higher ones to compose freely, and thus
            these special cases of $(n+k)$-categories are viewed as $n$-categories with $k$ products.
            Of course a $k$-tuply monoidal $n$-category is a special $k$-fold monoidal $n$-category. The specialization
            results from the definition(s) of $n$-category, all of which seem to include the axiom that
            the interchange 
            transformation between two ways of composing four higher morphisms along two different lower dimensions
            is required to be an isomorphism. As will be mentioned in the next section the  property of having 
            loop space nerves held by the $k$-fold categories relies on interchange transformations that are not 
            isomorphisms. If those transformations are indeed isomorphisms then the $k$-fold 1-categories do reduce to 
            the braided and symmetric
            1-categories of the periodic table. Whether this continues for higher dimensions, yielding for 
            example sylleptic monoidal
            2-categories as 3-fold 2-categories with interchange isomorphisms, is yet to be determined.
            
            A further refinement of higher categories is to require all morphisms to have inverses. These special cases
            are referred to as $n$-groupoids, and since their nerves are simpler to describe it has been long known
            that they model homotopy $n$-types. A homotopy $n$-type is a topological space 
            $X$ for which $\pi_k(X)$ is trivial
            for all $k>n.$ Thus the homotopy $n$-types are clasified by $\pi_n.$ It has been suggested that
            a key requirement for the eventual accepted definition of $n$-category is that a $k$-tuply monoidal
            $n$-groupoid be associated functorially (by a nerve) to a topological space which is a  
            homotopy $n$-type $k$-fold loop space. \cite{Baez1} The loop degree will be precise for $k<n+1,$ but for $k>n$ the 
            associated homotopy $n$-type will be an infinite loop space. This last statement is a consequence of 
            the stabilization hypothesis , which states that there should be a left adjoint to 
            forgetting monoidal structure
            that is an equivalence of $(n+k+2)$ categories between $k$-tuply monoidal $n$-categories and $(k+1)$-tuply
            monoidal $n$-categories for $k>n+1.$ For the case of $n=1$ if the 
            interchange transformations are isomorphic then a $k$-fold (and $k$-tuply) monoidal 1-category
            is equivalent to
            a symmetric category for $k>2.$
            With these facts in mind it is clear that if we wish to precisely model homotopy $n$-type $k$-fold loop spaces
            for $k>n$ then we need to consider $k$-fold as well as $k$-tuply monoidal $n$-categories.
            This paper is part of an embrionic effort in that direction. 
            
            Since a loop space can be efficiently described as an operad algebra, it is not surprising that there are 
            several existing definitions of $n$-category that utilize operad actions. These definitions fall into two main
            classes: those that define an $n$-category as an algebra of a higher order operad, and those that
            achieve an inductive definition using classical operads in symmetric monoidal categories to 
            parameterize iterated enrichment. The first class of definitions is typified by Batanin and Leinster.
            \cite{bat},\cite{lst}
            The
            former author defines monoidal globular categories in which interchange transformations are 
            isomorphisms and which thus resemble free strict $n$-categories.
             Globular operads live in these, and take all sorts of pasting diagrams
            as input types, as opposed to just a string of objects as in the case of classical operads.
            The binary composition in an $n$-category derives from the action of a certain one  
            of these globular operads.
            Leinster expands this concept to describe $n$-categories with unbiased composition of any number
            of cells.
            The second class of definitions is typified by the works of Trimble and May. 
            \cite{may2}, \cite{trimble}
            The former 
            parameterizes iterated enrichment with
            a series of operads in $(n-1)$-Cat achieved by taking the fundamental $(n-1)$-groupoid
            of the $k$th component of the topological path composition operad $E.$ 
            The latter begins with an $A_{\infty}$ operad in a symmetric
            monoidal category ${\cal V}$ and requires his enriched categories to be tensored over ${\cal V}$ so
            that the iterated enrichment always refers to the same original operad.
            
            Iterated enrichment over $n$-fold categories is described in \cite{forcey1} and \cite{forcey2}.
            We would like to define $n$-fold 
            operads in $n$-fold monoidal categories in a way that is 
            consistent with the spirit of Batanin's globular operads,
            and with the eventual goal of using them to weaken enrichment over $n$-fold categories
            in a way that is in the spirit of Trimble.
            This program carries with it the promise of characterizing $k$-fold loop spaces with homotopy $n$-type
            for all $n,k.$
            
            This paper comprises a naive beginning, illustrated with a very basic set of examples that we hope
            will help clarify the definitions. First we present the definition of $n$-fold monoidal
            category and go over a collection of related examples from semigroup theory. The examples of most
            visual value are from the combinatorial theory of tableau shapes. Secondly
            we present constructive definitions of $n$-fold operads and their algebras in an iterated monoidal category.
 We discuss examples of these that live in the previously described categories. Finally we describe 
 the $(n-2)$-fold monoidal category of $n$-fold ${\cal V}$-operads. 
 
 A more abstract approach for future consideration would begin by finding an equivalent definition of 
 $n$-fold operad in terms of monoids in a certain category. Then the full abstraction would be to
 find an equivalent definition in the language of Weber, where an operad lives within a 
 monoidal pseudo algebra of a 2-monad. \cite{web} This latter is a general notion of operad which includes as instances
 both classical and higher
 operads.

            
            
    
    \clearpage
          \newpage
        \MySection{$k$-fold monoidal categories}
      This sort of category was
    developed and defined in \cite{Balt}. The authors describe its structure as arising from its description
    as a monoid in the
        category of $(k-1)$-fold monoidal categories. Here is that definition altered only slightly 
    to make visible the coherent associators as in \cite{forcey1}. In that paper I describe its structure as 
    arising from its description
    as a tensor object in the
        category of $(k-1)$-fold monoidal categories.
        
\begin{definition}\label{iterated} An $n${\it -fold monoidal category} is a category ${\cal V}$
          with the following structure. 
          \begin{enumerate}
          \item There are $n$ distinct multiplications
          $$\otimes_1,\otimes_2,\dots, \otimes_n:{\cal V}\times{\cal V}\to{\cal V}$$
          for each of which the associativity pentagon commutes
          
          \noindent
  		          \begin{center}
  	          \resizebox{4.5in}{!}{
          $$
          \xymatrix@R=45pt@C=-35pt{
          &((U\otimes_i V)\otimes_i W)\otimes_i X \text{ }\text{ }
          \ar[rr]^{ \alpha^{i}_{UVW}\otimes_i 1_{X}}
          \ar[dl]^{ \alpha^{i}_{(U\otimes_i V)WX}}
          &\text{ }\text{ }\text{ }\text{ }\text{ }&\text{ }\text{ }(U\otimes_i (V\otimes_i W))\otimes_i X
          \ar[dr]^{ \alpha^{i}_{U(V\otimes_i W)X}}&\\
          (U\otimes_i V)\otimes_i (W\otimes_i X)
          \ar[drr]^{ \alpha^{i}_{UV(W\otimes_i X)}}
          &&&&U\otimes_i ((V\otimes_i W)\otimes_i X)
          \ar[dll]^{ 1_{U}\otimes_i \alpha^{i}_{VWX}}
          \\&&U\otimes_i (V\otimes_i (W\otimes_i X))&&&
          }
          $$
          }
  		                    \end{center}
  	                    
          
          ${\cal V}$ has an object $I$ which is a strict unit
          for all the multiplications.
          \item For each pair $(i,j)$ such that $1\le i<j\le n$ there is a natural
          transformation
          $$\eta^{ij}_{ABCD}: (A\otimes_j B)\otimes_i(C\otimes_j D)\to
          (A\otimes_i C)\otimes_j(B\otimes_i D).$$
          \end{enumerate}
          These natural transformations $\eta^{ij}$ are subject to the following conditions:
          \begin{enumerate}
          \item[(a)] Internal unit condition: 
          $\eta^{ij}_{ABII}=\eta^{ij}_{IIAB}=1_{A\otimes_j B}$
          \item[(b)] External unit condition:
          $\eta^{ij}_{AIBI}=\eta^{ij}_{IAIB}=1_{A\otimes_i B}$
          \item[(c)] Internal associativity condition: The following diagram commutes
          
          \noindent
	  	            	          \begin{center}
	      	          \resizebox{5in}{!}{
	     $$
            \diagram
            ((U\otimes_j V)\otimes_i (W\otimes_j X))\otimes_i (Y\otimes_j Z)
            \xto[rrr]^{\eta^{ij}_{UVWX}\otimes_i 1_{Y\otimes_j Z}}
            \ar[d]^{\alpha^i}
            &&&\bigl((U\otimes_i W)\otimes_j(V\otimes_i X)\bigr)\otimes_i (Y\otimes_j Z)
            \dto^{\eta^{ij}_{(U\otimes_i W)(V\otimes_i X)YZ}}\\
            (U\otimes_j V)\otimes_i ((W\otimes_j X)\otimes_i (Y\otimes_j Z))
            \dto^{1_{U\otimes_j V}\otimes_i \eta^{ij}_{WXYZ}}
            &&&((U\otimes_i W)\otimes_i Y)\otimes_j((V\otimes_i X)\otimes_i Z)
            \ar[d]^{\alpha^i \otimes_j \alpha^i}
            \\
            (U\otimes_j V)\otimes_i \bigl((W\otimes_i Y)\otimes_j(X\otimes_i Z)\bigr)
            \xto[rrr]^{\eta^{ij}_{UV(W\otimes_i Y)(X\otimes_i Z)}}
            &&& (U\otimes_i (W\otimes_i Y))\otimes_j(V\otimes_i (X\otimes_i Z))
            \enddiagram
            $$
           }
	            	                    \end{center}
  
           \item[(d)] External associativity condition: The following diagram commutes
            
            \noindent
  	            	          \begin{center}
      	          \resizebox{5in}{!}{ 
             $$
             \diagram
            ((U\otimes_j V)\otimes_j W)\otimes_i ((X\otimes_j Y)\otimes_j Z)
            \xto[rrr]^{\eta^{ij}_{(U\otimes_j V)W(X\otimes_j Y)Z}}
            \ar[d]^{\alpha^j \otimes_i \alpha^j}
            &&& \bigl((U\otimes_j V)\otimes_i (X\otimes_j Y)\bigr)\otimes_j(W\otimes_i Z)
            \dto^{\eta^{ij}_{UVXY}\otimes_j 1_{W\otimes_i Z}}\\
            (U\otimes_j (V\otimes_j W))\otimes_i (X\otimes_j (Y\otimes_j Z))
            \dto^{\eta^{ij}_{U(V\otimes_j W)X(Y\otimes_j Z)}}
            &&&((U\otimes_i X)\otimes_j(V\otimes_i Y))\otimes_j(W\otimes_i Z)
            \ar[d]^{\alpha^j}
            \\
            (U\otimes_i X)\otimes_j\bigl((V\otimes_j W)\otimes_i (Y\otimes_j Z)\bigr)
            \xto[rrr]^{1_{U\otimes_i X}\otimes_j\eta^{ij}_{VWYZ}}
            &&& (U\otimes_i X)\otimes_j((V\otimes_i Y)\otimes_j(W\otimes_i Z))
            \enddiagram
            $$
          }
	           	                    \end{center}
   
          \item[(e)] Finally it is required  for each triple $(i,j,k)$ satisfying
          $1\le i<j<k\le n$ that
          the giant hexagonal interchange diagram commutes.
         \end{enumerate}
          
          \noindent
  		          \begin{center}
  	          \resizebox{6in}{!}{
          $$
          \xymatrix@C=-118pt{
          &((A\otimes_k A')\otimes_j (B\otimes_k B'))\otimes_i((C\otimes_k C')\otimes_j (D\otimes_k D'))
          \ar[ddl]|{\eta^{jk}_{AA'BB'}\otimes_i \eta^{jk}_{CC'DD'}}
          \ar[ddr]|{\eta^{ij}_{(A\otimes_k A')(B\otimes_k B')(C\otimes_k C')(D\otimes_k D')}}
          \\\\
          ((A\otimes_j B)\otimes_k (A'\otimes_j B'))\otimes_i((C\otimes_j D)\otimes_k (C'\otimes_j D'))
          \ar[dd]|{\eta^{ik}_{(A\otimes_j B)(A'\otimes_j B')(C\otimes_j D)(C'\otimes_j D')}}
          &&((A\otimes_k A')\otimes_i (C\otimes_k C'))\otimes_j((B\otimes_k B')\otimes_i (D\otimes_k D'))
          \ar[dd]|{\eta^{ik}_{AA'CC'}\otimes_j \eta^{ik}_{BB'DD'}}
          \\\\
          ((A\otimes_j B)\otimes_i (C\otimes_j D))\otimes_k((A'\otimes_j B')\otimes_i (C'\otimes_j D'))
          \ar[ddr]|{\eta^{ij}_{ABCD}\otimes_k \eta^{ij}_{A'B'C'D'}}
          &&((A\otimes_i C)\otimes_k (A'\otimes_i C'))\otimes_j((B\otimes_i D)\otimes_k (B'\otimes_i D'))
          \ar[ddl]|{\eta^{jk}_{(A\otimes_i C)(A'\otimes_i C')(B\otimes_i D)(B'\otimes_i D')}}
          \\\\
          &((A\otimes_i C)\otimes_j (B\otimes_i D))\otimes_k((A'\otimes_i C')\otimes_j (B'\otimes_i D'))
          }
          $$
          }
  		                    \end{center}
  	                    
          
          
        \end{definition}
  
     
     The authors of \cite{Balt} remark that a symmetric monoidal category is $n$-fold monoidal for all $n$.
     This they demonstrate by letting
        $$\otimes_1=\otimes_2=\dots=\otimes_n=\otimes$$
        and defining (associators added by myself)
        $$\eta^{ij}_{ABCD}=\alpha^{-1}\circ (1_A\otimes \alpha)\circ (1_A\otimes (c_{BC}\otimes 1_D))\circ (1_A\otimes \alpha^{-1})\circ \alpha$$
        for all $i<j$. Here $c_{BC}: B\otimes C \to C\otimes B$ is the symmetry natural transformation.

\newpage   
\MySection{Examples of iterated monoidal categories} 
\begin{enumerate}
\item
We start with an example of a symmetric monoidal category. Given a totally ordered set $G$ with a smallest element $e \in G$, 
then we have an ordered semigroup structure where the semigroup product is max and the 2-sided unit is the
least element $e$. The elements of $G$ make up the objects of a strict monoidal category whose morphisms are
given by the ordering; there is only an arrow $a \to b$ if $a\le b.$ The ordered semigroup structure implies
the monoidal structure on morphisms, since if $a\le b$ and $a'\le b'$ 
 then $\max(a,a')\le\max(b,b').$ The category will also be denoted $G.$ 
If $G$ is such a set then by Seq($G$) we denote the infinite sequences $X_n$ of elements of $G$ for which
there exists a natural number $l(X)$ such that $k> l(X)$ implies $X_k = e$ and $X_{l(X)}\ne e.$ 
Under lexicographic ordering Seq($G$) is
in turn a totally ordered set with a least element. The latter is the sequence 0 where $0_n = e$ for all $n.$
We let $l(0) = 0.$
The lexicographic order
means that $A \le B$ if either $A_k = B_k$ for all $k$ or there is a natural number $n=n_{AB}$ such that 
$A_k = B_k$ for all $k < n$, and such that $A_{n} < B_{n}.$ 

The ordering is easily shown to be reflexive, transitive, and antisymmetric. See for instance \cite{Schrod}
where the case of lexicographic ordering of $n$-tuples of natural numbers is dicussed. In our case we will need
to modify the proof by always making comparisons of $\max(l(A),l(B))$-tuples. 
\item
As a category Seq($G$) is 2-fold monoidal since 
we can demonstrate two interchanging products: max using the lexicographic order-- $A\otimes_1 B = \max(A,B);$ 
and the piecewise max-- $(A \otimes_2 B)_n = \max(A_n, B_n).$ The latter is  more generally described as
the piecewise application of the ordered semigroup product from $G.$ Such a piecewise application
will always be in turn an ordered
semigroup product since $A_i \le B_i$ and $C_i \le D_i$
implies that $A_iC_i \le B_iD_i$. Seq($G$) is thus an example of 
a $2$-fold monoidal category that is formed from any totally ordered semigroup . 
\begin{theorem}\label{semi}
Given a totally ordered semigroup $\{H\le\}$
such that the
identity element $0 \in H$ is less than every other element then 
we have a 2-fold monoidal category whose
objects are elements of $H.$ 
\end{theorem}
{\bf Proof:~~~} Morphisms
are again given by the ordering. The products are given by
 $\max$ and the semigroup operation: $a\otimes_1b = \max(a,b)$ and $a\otimes_2b = ab$. 
 The shared two-sided unit for these products is 
 the identity element $0.$ The products are both strictly associative and functorial since if $a\le b$ and $a'\le b'$ 
 then $aa'\le bb'$ and $\max(a,a')\le\max(b,b').$
 The interchange natural tranformations exist since $\max(ab,cd)\le\max(a,c)\max(b,d).$ This last theorem is easily seen by checking the four
 possible cases: $\{c\le a, d\le b\};~~\{c\le a, b\le d\};~~\{a\le c, d\le b\};~~\{a\le c, b\le d\};$ or by the 
 quick argument that
 
 $
   a \le max(a,c) \text{ and } b \le max(b,d)  \text{ so } $
   
              $  ab \le max(a,c)max(b,d)   \text{ and similarly } $
                
              $  cd \le max(a,c)max(b,d).
$
 
 The internal and external unit and associativity conditions of Definition~\ref{iterated} are all satisfied due to the
 fact that there is only one morphism between two objects. 
  Besides Seq($G$) examples
 of such semigroups as in Theorem~\ref{semi} are the nonnegative integers under addition
 and the braids on $n$ strands with only right-handed crossings. Further examples are found in the papers on
 semirings and idempotent mathematics, such as \cite{LitSob} and its references as well as on
 the related concept of tropical mathematics, such as \cite{Sturm} and its references. Semirings that arise in these
 two areas of study usually require some 
 translation to yield 2-fold monoidal categories since they typically have a zero and a unity, $0 \ne 1.$ 
 \item
 There is also an alternative
 to the piecewise max on the objects of Seq($G$)-- concatenation of sequences. This is a valid semigroup
 operation and thus monoidal product since it clearly preserves the lexicographic order.
 Notice that concatenation, like braid composition, is a non-symmetric example. A non-example,
 however, is revealed in an attempt to have both the piecewise max and the concatenation at once in
 an iterated monoidal category. The required interchange morphisms between
 concatenation and piecewise max do not exist. 
 \item
 The step taken to create Seq($G$) can be iterated to create $n$-fold monoidal categories for all $n.$ 
 In general, if we have a $k$-fold monoidal category $C$ with morphisms ordering of objects and the
common unit and smallest element $I$ such that $A\in C \Rightarrow I\le A$ then:

\begin{tabular}{ll}
1.&we can form a $(k+1)$-fold monoidal category $\hat{C}$ with $\hat{\otimes}_1 = \max$ and $\hat{\otimes}_i = \otimes_{i-1}$\\
2.&we can form the $(k+2)$-fold monoidal Seq$(C)$ with objects ordered\\ 
&lexicographically.
\end{tabular}
Now the new $\otimes_1$ of Seq($C$) is the maximum of sequences with
respect to the lexicographic order, and the new $\otimes_2 ... \otimes_{k+2}$ are the piecewise products based 
respectively on 
$\hat{\otimes}_1 ... \hat{\otimes}_{k+1}$ in $\hat{C}.$
\item   
   Even more structure is found in examples 
   with a natural geometric representation which allows
   the use of addition in each product. One such category is that whose objects are Ferrers diagrams,
   by which we mean the shapes
   of Young tableaux. These
   can be presented by a decreasing sequence of nonnegative integers in two ways: the sequence that gives the heights
   of the columns or the sequence that gives the lengths of the rows. We let $\otimes_2$ be the product which
   adds the heights of columns of two tableaux, $\otimes_1$
   adds the length of rows. 
   $$
   A = \xymatrix@W=1.8pc @H=1.8pc @R=0pc @C=0pc @*[F-]{~&~&~&~\\~}
   \text{ and } B = \xymatrix@W=1.8pc @H=1.8pc @R=0pc @C=0pc @*[F-]{~&~\\~\\~}
   $$
   $$
   \text{then } A\otimes_1 B = \xymatrix@W=1.8pc @H=1.8pc @R=0pc @C=0pc @*[F-]{~&~&~&~&~&~\\~&~\\~}
   \text{ and } A\otimes_2 B = \xymatrix@W=1.8pc @H=1.8pc @R=0pc @C=0pc @*[F-]{~&~&~&~\\~&~\\~\\~\\~}
   $$   
   There are several possibilities for morphisms. We can
   create a category equivalent to the non-negative integers in the previous example by pre-ordering the tableaux by height. 
   Here the height $h(A)$ of the tableau is the number of boxes in its leftmost column, and  we say $A \le B$ if $h(A)\le h(B)$. 
   Two tableaux with the same height are isomorphic objects, and the one-column stacks form both a full subcategory 
   and a  skeleton of the height preordered category.
   Everything works as for the previous example of natural numbers since $h(A\otimes_2 B) = h(A) +h(B)$ and $h(A\otimes_1 B) = \max(h(A), h(B)).$
   
   Since we are working with sequences there are also inherited max products as dicussed in the first example.
   The new max with respect to the height preordering is defined as $\max(A,B) = A$ if $B\le A$ and $ = B$ otherwise.
   The piecewise application of max with respect to the ordering of natural numbers results in taking a union
   of the two tableau shapes. 
   Above, $\max(A,B) = B$ and the piecewise max becomes
   $$
   \xymatrix@W=1.8pc @H=1.8pc @R=0pc @C=0pc @*[F-]{~&~&~&~\\~\\~}
   $$
   In the height preordered category both of these latter products are equivalent to the horizontal composition $\otimes_1.$
\item   
Alternatively we can work with the totally ordered structure of the 
tableau shapes given by lexicographic ordering. 
For now we consider the presentation in terms of a monotone decreasing sequence of columns, and regard
an empty column as a height zero column. Every tableau $A$ considered
as an infinite sequence has all zero height columns after some finite natural number $l(A)$. 
Now the four products just described in the second example above still exist, but are truly
distinct. 
The overall max with respect to the lexicographic order is thus $\otimes_1.$ Piecewise max is
$\otimes_2.$ Horizontal addition is $\otimes_3$ and vertical addition  
(piecewise addition of sequence terms) is $\otimes_4.$ 
For comparison purposes here is an example of the four possible products.
For objects
$$
A=\xymatrix@W=1.8pc @H=1.8pc @R=0pc @C=0pc @*[F-]{~&~&~\\~\\~}
\text{ and } B = \xymatrix@W=1.8pc @H=1.8pc @R=0pc @C=0pc @*[F-]{~&~\\~&~}
$$
$$
\text{then } A\otimes_1 B = \xymatrix@W=1.8pc @H=1.8pc @R=0pc @C=0pc @*[F-]{~&~&~\\~\\~}
\text{ and } A\otimes_2 B = \xymatrix@W=1.8pc @H=1.8pc @R=0pc @C=0pc @*[F-]{~&~&~\\~&~\\~}
$$
$$
 A\otimes_3 B = \xymatrix@W=1.8pc @H=1.8pc @R=0pc @C=0pc @*[F-]{~&~&~&~&~\\~&~&~\\~}
\text{ and } A\otimes_4 B = \xymatrix@W=1.8pc @H=1.8pc @R=0pc @C=0pc @*[F-]{~&~&~\\~&~\\~&~\\~\\~}
$$  
\begin{theorem}\label{modseq}
Tableau shapes, lexicographic ordering, and $\otimes_1, \otimes_2, \otimes_3, \otimes_4$
as described above form a 4-fold monoidal category.
\end{theorem}
{\bf Proof:~~~}
By previous discussion
then the tableau shapes with $\otimes_1, \otimes_2, \otimes_4$ form a 3-fold monoidal category
called Seq({\bf N}).
To see that with the additional $\otimes_3$ this becomes a valid 4-fold monoidal category  
we need to check first 
that horizontal addition is functorial with respect to morphisms (defined as the $\le$ relations 
of the lexicographic ordering.)  
Note that the horizontal product of tableau shapes $A$ and $C$ can be described as
a reorganization of all the columns of both $A$ and $C$ into a new tableau shape made up of those columns in
 descending order of height. 
 Especially when speaking of general sequences we'll call this product sorting.  The cases where
 $A=B$ or $C=D$ are easy. For example let $A_k = B_k$ for all $k$ and
$C_k = D_k$ for all $k < n_{CD}$ Thus the columns from the copies of, for instance $A$ in $A\otimes_1 C$ and 
$A\otimes_1 D$ fall into the same final spot under the sortings right up to the critical location, so
if $C \le D$, then $A\otimes_1 C \le A\otimes_1 D.$
Similarly, it is clear that $A \le B$ implies $(A \otimes_1 D) \le (B \otimes_1 D).$ 
Hence if $A\le B$ and $C\le D$, then
        $A\otimes_1 C \le A\otimes_1 D \le B\otimes_1 D$ which by transitivity gives us our desired property. 




%Now then we assume strict inequalities $A < B$ and $C<D$ 
%and observe the process of creating $A\otimes_1 C $ and $ A\otimes_1 D$ step by step. First put in descending order
%all columns $A_k $ for $k \le n_{AB}$ and $C_j $ for  $j \le n_{CD}.$ We have a sequence of numbers that is precisely
%the same as the descending order listing of all columns $B_k$ for $k \le n_{AB}$ and
%$D_k$ for $k \le n_{CD}.$ Now we consider the next step, of inserting the next columns from our tableau
% shapes into their respective products. We are putting $A_{n_{AB}}$ and $C_{n_{CD}}$ into their
% ordered locations in the descending sequence constructed thus far,
% and putting $B_{n_{AB}}$ and $D_{n_{CD}}$ into an identical sequence. Let $x=\max(A_{n_{AB}},C_{n_{CD}})$
% and $y=\max(B_{n_{AB}},D_{n_{CD}}).$ Then since $A_{n_{AB}} <B_{n_{AB}}$ and $C_{n_{CD}} < D_{n_{CD}}$
% we have that $x<y$. Thus $y$ will be inserted into our sequence of columns at a location closer to
% the beginning than will $x.$ Therefore since $y$ is inserted at a location where it is strictly greater than the 
% entry it is inserted imediately prior to, $y$ will also be strictly greater than the corresponding term in the identical sequence 
% into which we inserted $x.$ Here is a quick explanatory diagram where the blanks represent the two identical sequences.
% 
% $$
%\xymatrix@R=9pt{
%\_\ar@{=}[d]&\_\ar@{=}[dr]&\_\ar@{=}[dr]&x&\_\ar@{=}[d]&\_\ar@{=}[d]&\_\ar@{=}[d]\\
%\_&y\ar@{}[u]|{\Lambda}\ar@{}[r]|{>}&\_&\_&\_&\_&\_
%}
%$$
%
%We conclude that
%$A\otimes_1 C \le B\otimes_1 D$ since the remaining work to be done in constructing the products consists of inserting 
%column height values that will always be smaller than $x$ and $y$ respectively and so will not affect the critical
%location.
 
 Next we check that our interchange transformations will always exist. $\eta^{1j}$ exists by
 the proof of Theorem~\ref{semi} for $j=2,3,4$ since the higher products all respect morphisms(ordering) and
 are thus ordered semigroup operations. We need to check for existence of $\eta^{23}, \eta^{24},$ and $\eta^{34}.$
 
 First for existence of $\eta^{23}$ we need to show that
   $(A \otimes_3 B)\otimes_2(C \otimes_3 D) \le (A \otimes_2 C)\otimes_3 (B \otimes_2 D).$
      We actually show more generally that for any two sequences $a$ and $b$ that 
   $\max_{piecewise}(\sort(a),\sort(b)) \le \sort(\max_{piecewise}(a,b)).$
where we are sorting greater to smaller and taking piecewise maximums. 
To prove the original question we consider the special case of two sequences formed by letting $a$ be $A$ followed by 
$C$ and letting $b$ be $B$ followed by $D$. By ``followed by'' we mean padded by zeroes so that
piecewise addition of $a$ and $b$ results in piecewise addition of $A$ and $B$, and respectively $C$ and $D.$
Thus in the original question $l(a) = \max(l(A),l(B)) + l(C) $ and $l(b) = \max(l(A),l(B)) + l(D).$
Recall that $a_k = 0$ for $k>l(a)$.

Consider the right hand side of the inequality, where we start with the two sequences and take the 
piecewise maximum of the corresponding pairs of elements before sorting. We can metamorphose this
into the left hand side in stages by using an algorithm to sort $a$ and $b$. Note that if $a$ and $b$ are
already sorted, the inequality becomes an equality.
For our algorithm we choose parallel bubble sorts. These consist
of a series of passes through the sequences comparing $a_n$ and $a_{n+1}$ and  comparing $b_n$ and $b_{n+1}$
simultaneously. If the two elements of a given sequence are not already in decreasing order we 
switch their places. We claim that such an operation always results in a lexicographically smaller
sequence after taking piecewise max and sorting. If both the elements of $a$ and of $b$ are switched, or if neither,
then the result is unaltered. Therefore without loss of generality we assume that $a_n < a_{n+1}$
and that $b_{n+1} \le b_n.$ Then we compare the original
result of sorting after taking the piecewise max and the same but after the switching of $a_n$ and
$a_{n+1}.$ If $b_n \le a_n$ and $b_{n+1} \le a_{n+1}$ then 
$\max(a_n,b_n) = \max(a_n, b_{n+1})$ and $\max(a_{n+1}, b_{n+1}) = \max(a_{n+1}, b_n).$ If $a_n \le b_n$ 
and $a_{n+1} \le b_{n+1}$  then
$\max(a_n,b_n) = \max(a_{n+1}, b_n)$ and $\max(a_{n+1}, b_{n+1}) = \max(a_n, b_{n+1}).$
 Since neither of these cases lead to any change in the final result,
 we may assume $a_n \le b_n$ and $b_{n+1} \le a_{n+1}$, and check two sub-cases:
First $a_{n+1} \le b_n$ implies that
$\max(a_n,b_n) = \max(a_{n+1}, b_n)$ and $\max(a_n, b_{n+1}) \le \max(a_{n+1}, b_{n+1}).$ Secondly $b_n \le a_{n+1}$
implies that $\max(a_n, b_{n+1}) \le max(a_n,b_n)$ and $\max(a_{n+1}, b_{n+1}) = \max(a_{n+1}, b_n).$ 

Thus after taking piecewise max
and sorting the new result is indeed smaller lexicographically. 

Thus since each move of the parallel bubble sort results in a smaller expression  and the moves
eventually result in $(A \otimes_3 B)\otimes_2(C \otimes_3 D)$, we have $(A \otimes_3 B)\otimes_2(C \otimes_3 D) \le (A \otimes_2 C)\otimes_3 (B \otimes_2 D).$  
  
Secondly the existence of $\eta^{24}$ is clear since the interchange between max and + implies the interchange between 
their piecewise application. Thus we have $(A \otimes_4 B)\otimes_2(C \otimes_4 D) \le (A \otimes_2 C)\otimes_4 (B \otimes_2 D).$
      
   
Finally we need to check that $(A \otimes_4 B)\otimes_3(C \otimes_4 D) \le(A \otimes_3 C)\otimes_4 (B \otimes_3 D).$  
We actually show more generally that for any two sequences $a$ and $b$ of total length
that $Y= \sort(a+b)\le \sort(a)+\sort(b) =X$ where $+$ is the piecewise addition.
To prove the original question we make the two sequences by letting $a$ be $A$ followed by 
$C$ and letting $b$ be $B$ followed by $D$ as in the previous proof.

Consider $Y$, where we start with the two sequences and add them 
piecewise before sorting. We can metamorphose this
into $X$ in stages by using an algorithm to sort $a$ and $b$. Note that if $a$ and $b$ are
already sorted, the inequality becomes an equality.
For our algorithm we again use parallel bubble sorts.  We claim that switching consecutive 
sequence elements into order
always results in a lexicographically larger
sequence after adding piecewise and sorting. If both the elements of $a$ and of $b$ are switched, or if neither,
then the result is unaltered. Therefore without loss of generality we assume that $a_n < a_{n+1}$
and that $b_{n+1} < b_n.$ Then we compare the original
result of sorting after adding and the same but after the switching of $a_n$ and
$a_{n+1}.$ It is simplest to note that the new result includes $a_{n+1} + b_n,$ which is larger than
both $a_n + b_n$ and $a_{n+1} + b_{n+1}.$ 

So after adding
and sorting the new result is indeed larger lexicographically. 
Thus since each move of the parallel bubble sort results in a larger expression  and the moves
eventually result in $(A \otimes_3 C)\otimes_4 (B \otimes_3 D)$, we have $(A \otimes_4 B)\otimes_3(C \otimes_4 D) \le(A \otimes_3 C)\otimes_4 (B \otimes_3 D).$

%We use strong induction on total length. The inequality is trivially true for total length of 1. 
%For a total length of $n+1$ we note the 
%greatest pair of numbers which are in the same location in $\sort(a)$ and $\sort(b)$ but are in location $a_i$
%and $b_j$, $i \ne j$, respectively in $a$ and $b$. Without loss of generality we assume $i>j$ and form 
%$b' = b_1,\dots,b_{j-1},\hat{b_j},\dots,\hat{b_{i-1}},b_i,\dots,b_{l(b)},0\dots$
%by deleting entries $b_j ... b_{i-1}$ from $b.$
%Then let $X' = \sort(a)+\sort(b')$ and $Y' = \sort(a+b').$ By construction $X'<X$ and $Y'>Y.$ By induction
%since $l(a)+l(b') = n+1 - (i-j) \le n$ then $Y' \le X'.$ So $Y<Y'\le X'<X$ and we have the required inequality. 


% Consider the results of these two operations, letting
%   $X=(A \otimes_2 B)\otimes_1(C \otimes_2 D)$
% and $Y(A \otimes_1 C)\otimes_2 (B \otimes_1 D).$ Assume that $Y<X.$ Then $Y_k = X_k$ for $k<n=n_{YX}$ and $Y_n < X_n.$
% For $k<n$ let $Y_k = X_k = p_k + q_k$ where $p_k$ is the height of a column in $A$ or $C$ and $q_k$ is the height of a column
% in $B$ or $D.$ Then $Y_n = y+y'$ where $y = \max\{A\bigcup C - \{p_k\}_{k=1}^{n-1}\}$, the 
% largest column in $A$ or $C$ excepting the columns $p_k,$ and $y' = \max\{B\bigcup D-\{q_k\}_{k=1}^{n-1}\},$ the 
% largest column in $B$ or $D$ excepting the columns $q_k.$
% Now $X_n = x + x'$ for some pair $x,x'$ where $x$ is the height of a column in $A$ or $C$ excepting the columns $p_k$ 
% and $x'$ is the height of a column
% in $B$ or $D$ excepting the columns $q_k.$ We have that either $y<x$ or $y'<x'$ since $Y_n < X_n.$ But this 
% contradicts the fact that $y$ and $y'$ both are maximal. Therefore $(A \otimes_2 B)\otimes_1(C \otimes_2 D) \le 
% (A \otimes_1 C)\otimes_2 (B \otimes_1 D)$.




\item
Notice that in the last two sections of the proof of Theorem~\ref{modseq}, the 
product of two sequences given by sorting their elements
behaves differently with respect to piecewise max and piecewise +. In general for a category of decreasing sequences
the sorting given by $\otimes_k$ will have an interchange with a piecewise operation ${(A\otimes_j B)}_n = A_n*B_n, j < k$
if $ A_n \le B_n $ 
implies $A_n*B_n = B_n.$ Also $\otimes_k$ will have an interchange with a  piecewise operation ${(A\otimes_j B)}_n = A_n*B_n, j > k$
if $ A_n > B_n $ 
implies $A_n*C_n > B_n*C_n.$
This is important when attempting to
 iterate the construction.
 We can start with any totally ordered semigroup $\{G,\le, +\}$ such that the identity
0 is less than any other element and such that $ a > b $ 
implies $a+c > b+c$ for all $a,b,c \in G.$ 
We create a 4-fold monoidal category ModSeq$(G)$ with objects monotone decreasing 
finitely non-zero sequences of elements of $G$ and morphisms given by the lexicographic ordering. 
The products are as described for natural numbers in the previous example.  The common unit is the zero sequence.   
The proofs we have given in 
the previous example for $G = ${\bf N} are all still valid.  The objects of ModSeq($G$), sequences under lexicographic order,
 form four totally ordered semigroups
using each of the products respectively as the group operation. Thus we can form
four 3-fold monoidal categories Seq(ModSeq($G,*$)) based on each of these semigroup structures. 
Forming ModSeq(ModSeq($G$)) is only possible for the semigroup structures that use 
$\otimes_3$ or $\otimes_4$ from Theorem~\ref{modseq}, It would seem that we also could potentially use the objects of
ModSeq(ModSeq($G$)) in a 6-fold category. However, we must choose between two possible 5-fold monoidal categories
 of decreasing sequences (of terms which are themselves decreasing sequences).
The reason is that the product given by sorting two sequences of sequences by lexicographic order of their element sequences
does not interchange at all with the product given by taking piecewise--piecewise max, by which we mean
entrywise max, at the level of original semigroup elements. Either of these two as $\otimes_3$ however,
along with overall (lexicographic ordering of sequence of sequences) max as $\otimes_1$, piecewise lexicographic max as $\otimes_2$, 
piecewise sorting as $\otimes_4$, and piecewise--piecewise addition as $\otimes_5$,
do form 5-fold monoidal categories.

% The iteration continues. In general, if we have a $k$-fold monoidal category $C$ with ordering of objects and the
%common unit $I$ such that $A\in C \Rightarrow I\le A,$
% we can form the $(k+2)$-fold monoidal ModSeq$(C)$ with objects ordered lexicographically. 
%$\otimes_1$ and $\otimes_2$ are the is the recombining of sequences with
%respect to the underlying order, and $\otimes'_2 ... \otimes'_{k+1}$ are the component-wise products based 
%respectively on 
%$\otimes_1 ... \otimes_k$ in $C.$



For instance if ModSeq({\bf N}) is our category of tableau shapes
then ModSeq(ModSeq({\bf N})) has objects monotone decreasing sequences of tableau, which we can visualize along the
$z$-axis. For example, choosing the lexicographic sorting as $\otimes_3$, if:
$$
A=\xymatrix @W=1.0pc @H=1.0pc @R=1.0pc @C=1.0pc
 {*=0{}&*=0{}&*=0{}&*=0{}&*=0{}\ar@{-}[r]\ar@{-}[dl]&*=0{}\ar@{-}[r]\ar@{-}[dl]&*=0{}\ar@{-}[r]\ar@{-}[dl]&*=0{}\ar@{-}[r]\ar@{-}[dl]&*=0{}\ar@{-}[dl]
\\*=0{}&*=0{}&*=0{}&*=0{}\ar@{-}[r]\ar@{-}[dl]&*=0{}\ar@{-}[r]\ar@{-}[dl]&*=0{}\ar@{-}[r]\ar@{-}[dl]&*=0{}\ar@{-}[r]&*=0{}
\\*=0{}&*=0{}&*=0{}\ar@{-}[r]\ar@{-}[dl]&*=0{}\ar@{-}[r]\ar@{-}[dl]&*=0{}\ar@{-}[dl]
\\*=0{}&*=0{}\ar@{-}[r]\ar@{-}[dl]&*=0{}\ar@{-}[r]\ar@{-}[dl]&*=0{}
\\*=0{}\ar@{-}[r]&*=0{}
\\
*=0{}&*=0{}&*=0{}&*=0{}&*=0{}\ar@{-}[r]\ar@{-}[dl]&*=0{}\ar@{-}[r]\ar@{-}[dl]&*=0{}\ar@{-}[r]\ar@{-}[dl]&*=0{}\ar@{-}[r]\ar@{-}[dl]&*=0{}\ar@{-}[r]\ar@{-}[dl]&*=0{}\ar@{-}[r]\ar@{-}[dl]&*=0{}\ar@{-}[dl]
\\*=0{}&*=0{}&*=0{}&*=0{}\ar@{-}[r]\ar@{-}[dl]&*=0{}\ar@{-}[r]\ar@{-}[dl]&*=0{}\ar@{-}[r]\ar@{-}[dl]&*=0{}\ar@{-}[r]&*=0{}\ar@{-}[r]&*=0{}\ar@{-}[r]&*=0{}
\\*=0{}&*=0{}&*=0{}\ar@{-}[r]\ar@{-}[dl]&*=0{}\ar@{-}[r]\ar@{-}[dl]&*=0{}
\\*=0{}&*=0{}\ar@{-}[r]\ar@{-}[dl]&*=0{}\ar@{-}[dl]
\\*=0{}\ar@{-}[r]&*=0{}
\\
*=0{}&*=0{}&*=0{}&*=0{}&*=0{}\ar@{-}[r]\ar@{-}[dl]&*=0{}\ar@{-}[r]\ar@{-}[dl]&*=0{}\ar@{-}[r]\ar@{-}[dl]&*=0{}\ar@{-}[dl]
\\*=0{}&*=0{}&*=0{}&*=0{}\ar@{-}[r]\ar@{-}[dl]&*=0{}\ar@{-}[r]\ar@{-}[dl]&*=0{}\ar@{-}[r]\ar@{-}[dl]&*=0{}
\\*=0{}&*=0{}&*=0{}\ar@{-}[r]\ar@{-}[dl]&*=0{}\ar@{-}[r]\ar@{-}[dl]&*=0{}\ar@{-}[dl]
\\*=0{}&*=0{}\ar@{-}[r]&*=0{}\ar@{-}[r]&*=0{}
\\
*=0{}&*=0{}&*=0{}&*=0{}&*=0{}\ar@{-}[r]\ar@{-}[dl]&*=0{}\ar@{-}[r]\ar@{-}[dl]&*=0{}\ar@{-}[r]\ar@{-}[dl]&*=0{}\ar@{-}[r]\ar@{-}[dl]&*=0{}\ar@{-}[dl]
\\*=0{}&*=0{}&*=0{}&*=0{}\ar@{-}[r]&*=0{}\ar@{-}[r]&*=0{}\ar@{-}[r]&*=0{}\ar@{-}[r]&*=0{}
\\*=0{}}
\text{ and }B=\xymatrix @W=1.0pc @H=1.0pc @R=1.0pc @C=1.0pc
 {*=0{}&*=0{}&*=0{}&*=0{}\ar@{-}[r]\ar@{-}[dl]&*=0{}\ar@{-}[r]\ar@{-}[dl]&*=0{}\ar@{-}[dl]
\\*=0{}&*=0{}&*=0{}\ar@{-}[r]\ar@{-}[dl]&*=0{}\ar@{-}[r]\ar@{-}[dl]&*=0{}
\\*=0{}&*=0{}\ar@{-}[r]\ar@{-}[dl]&*=0{}\ar@{-}[dl]
\\*=0{}\ar@{-}[r]&*=0{}
\\
  *=0{}&*=0{}&*=0{}&*=0{}\ar@{-}[r]\ar@{-}[dl]&*=0{}\ar@{-}[r]\ar@{-}[dl]&*=0{}\ar@{-}[r]\ar@{-}[dl]&*=0{}\ar@{-}[dl]
\\*=0{}&*=0{}&*=0{}\ar@{-}[r]\ar@{-}[dl]&*=0{}\ar@{-}[r]\ar@{-}[dl]&*=0{}\ar@{-}[r]&*=0{}
\\*=0{}&*=0{}\ar@{-}[r]&*=0{}
\\
  *=0{}&*=0{}&*=0{}&*=0{}\ar@{-}[r]\ar@{-}[dl]&*=0{}\ar@{-}[r]\ar@{-}[dl]&*=0{}\ar@{-}[dl]
\\*=0{}&*=0{}&*=0{}\ar@{-}[r]&*=0{}\ar@{-}[r]&*=0{}
\\*=0{}}
$$
then $A\otimes_1 B = A$, $A\otimes_2 B = A$ and 
$$
A \otimes_3 B=\xymatrix @W=1.0pc @H=1.0pc @R=1.0pc @C=1.0pc
 {*=0{}&*=0{}&*=0{}&*=0{}&*=0{}\ar@{-}[r]\ar@{-}[dl]&*=0{}\ar@{-}[r]\ar@{-}[dl]&*=0{}\ar@{-}[r]\ar@{-}[dl]&*=0{}\ar@{-}[r]\ar@{-}[dl]&*=0{}\ar@{-}[dl]
\\*=0{}&*=0{}&*=0{}&*=0{}\ar@{-}[r]\ar@{-}[dl]&*=0{}\ar@{-}[r]\ar@{-}[dl]&*=0{}\ar@{-}[r]\ar@{-}[dl]&*=0{}\ar@{-}[r]&*=0{}
\\*=0{}&*=0{}&*=0{}\ar@{-}[r]\ar@{-}[dl]&*=0{}\ar@{-}[r]\ar@{-}[dl]&*=0{}\ar@{-}[dl]
\\*=0{}&*=0{}\ar@{-}[r]\ar@{-}[dl]&*=0{}\ar@{-}[r]\ar@{-}[dl]&*=0{}
\\*=0{}\ar@{-}[r]&*=0{}
\\
  *=0{}&*=0{}&*=0{}&*=0{}&*=0{}\ar@{-}[r]\ar@{-}[dl]&*=0{}\ar@{-}[r]\ar@{-}[dl]&*=0{}\ar@{-}[r]\ar@{-}[dl]&*=0{}\ar@{-}[r]\ar@{-}[dl]&*=0{}\ar@{-}[r]\ar@{-}[dl]&*=0{}\ar@{-}[r]\ar@{-}[dl]&*=0{}\ar@{-}[dl]
\\*=0{}&*=0{}&*=0{}&*=0{}\ar@{-}[r]\ar@{-}[dl]&*=0{}\ar@{-}[r]\ar@{-}[dl]&*=0{}\ar@{-}[r]\ar@{-}[dl]&*=0{}\ar@{-}[r]&*=0{}\ar@{-}[r]&*=0{}\ar@{-}[r]&*=0{}
\\*=0{}&*=0{}&*=0{}\ar@{-}[r]\ar@{-}[dl]&*=0{}\ar@{-}[r]\ar@{-}[dl]&*=0{}
\\*=0{}&*=0{}\ar@{-}[r]\ar@{-}[dl]&*=0{}\ar@{-}[dl]
\\*=0{}\ar@{-}[r]&*=0{}
\\
  *=0{}&*=0{}&*=0{}&*=0{}&*=0{}\ar@{-}[r]\ar@{-}[dl]&*=0{}\ar@{-}[r]\ar@{-}[dl]&*=0{}\ar@{-}[r]\ar@{-}[dl]&*=0{}\ar@{-}[dl]
\\*=0{}&*=0{}&*=0{}&*=0{}\ar@{-}[r]\ar@{-}[dl]&*=0{}\ar@{-}[r]\ar@{-}[dl]&*=0{}\ar@{-}[r]\ar@{-}[dl]&*=0{}
\\*=0{}&*=0{}&*=0{}\ar@{-}[r]\ar@{-}[dl]&*=0{}\ar@{-}[r]\ar@{-}[dl]&*=0{}\ar@{-}[dl]
\\*=0{}&*=0{}\ar@{-}[r]&*=0{}\ar@{-}[r]&*=0{}
\\
  *=0{}&*=0{}&*=0{}&*=0{}&*=0{}\ar@{-}[r]\ar@{-}[dl]&*=0{}\ar@{-}[r]\ar@{-}[dl]&*=0{}\ar@{-}[dl]
\\*=0{}&*=0{}&*=0{}&*=0{}\ar@{-}[r]\ar@{-}[dl]&*=0{}\ar@{-}[r]\ar@{-}[dl]&*=0{}
\\*=0{}&*=0{}&*=0{}\ar@{-}[r]\ar@{-}[dl]&*=0{}\ar@{-}[dl]
\\*=0{}&*=0{}\ar@{-}[r]&*=0{}
\\
  *=0{}&*=0{}&*=0{}&*=0{}&*=0{}\ar@{-}[r]\ar@{-}[dl]&*=0{}\ar@{-}[r]\ar@{-}[dl]&*=0{}\ar@{-}[r]\ar@{-}[dl]&*=0{}\ar@{-}[dl]
\\*=0{}&*=0{}&*=0{}&*=0{}\ar@{-}[r]\ar@{-}[dl]&*=0{}\ar@{-}[r]\ar@{-}[dl]&*=0{}\ar@{-}[r]&*=0{}
\\*=0{}&*=0{}&*=0{}\ar@{-}[r]&*=0{}
\\
  *=0{}&*=0{}&*=0{}&*=0{}&*=0{}\ar@{-}[r]\ar@{-}[dl]&*=0{}\ar@{-}[r]\ar@{-}[dl]&*=0{}\ar@{-}[r]\ar@{-}[dl]&*=0{}\ar@{-}[r]\ar@{-}[dl]&*=0{}\ar@{-}[dl]
\\*=0{}&*=0{}&*=0{}&*=0{}\ar@{-}[r]&*=0{}\ar@{-}[r]&*=0{}\ar@{-}[r]&*=0{}\ar@{-}[r]&*=0{}
\\
  *=0{}&*=0{}&*=0{}&*=0{}&*=0{}\ar@{-}[r]\ar@{-}[dl]&*=0{}\ar@{-}[r]\ar@{-}[dl]&*=0{}\ar@{-}[dl]
\\*=0{}&*=0{}&*=0{}&*=0{}\ar@{-}[r]&*=0{}\ar@{-}[r]&*=0{}
\\*=0{}}
A \otimes_4 B=\xymatrix @W=1.0pc @H=1.0pc @R=1.0pc @C=1.0pc
  {*=0{}&*=0{}&*=0{}&*=0{}&*=0{}\ar@{-}[r]\ar@{-}[dl]&*=0{}\ar@{-}[r]\ar@{-}[dl]&*=0{}\ar@{-}[r]\ar@{-}[dl]&*=0{}\ar@{-}[r]\ar@{-}[dl]&*=0{}\ar@{-}[r]\ar@{-}[dl]&*=0{}\ar@{-}[r]\ar@{-}[dl]&*=0{}\ar@{-}[dl]
\\*=0{}&*=0{}&*=0{}&*=0{}\ar@{-}[r]\ar@{-}[dl]&*=0{}\ar@{-}[r]\ar@{-}[dl]&*=0{}\ar@{-}[r]\ar@{-}[dl]&*=0{}\ar@{-}[r]\ar@{-}[dl]&*=0{}\ar@{-}[r]&*=0{}\ar@{-}[r]&*=0{}
\\*=0{}&*=0{}&*=0{}\ar@{-}[r]\ar@{-}[dl]&*=0{}\ar@{-}[r]\ar@{-}[dl]&*=0{}\ar@{-}[r]\ar@{-}[dl]&*=0{}\ar@{-}[dl]
\\*=0{}&*=0{}\ar@{-}[r]\ar@{-}[dl]&*=0{}\ar@{-}[r]\ar@{-}[dl]&*=0{}\ar@{-}[r]&*=0{}
\\*=0{}\ar@{-}[r]&*=0{}
\\
  *=0{}&*=0{}&*=0{}&*=0{}&*=0{}\ar@{-}[r]\ar@{-}[dl]&*=0{}\ar@{-}[r]\ar@{-}[dl]&*=0{}\ar@{-}[r]\ar@{-}[dl]&*=0{}\ar@{-}[r]\ar@{-}[dl]&*=0{}\ar@{-}[r]\ar@{-}[dl]&*=0{}\ar@{-}[r]\ar@{-}[dl]&*=0{}\ar@{-}[r]\ar@{-}[dl]&*=0{}\ar@{-}[r]\ar@{-}[dl]&*=0{}\ar@{-}[r]\ar@{-}[dl]&*=0{}\ar@{-}[dl]
\\*=0{}&*=0{}&*=0{}&*=0{}\ar@{-}[r]\ar@{-}[dl]&*=0{}\ar@{-}[r]\ar@{-}[dl]&*=0{}\ar@{-}[r]\ar@{-}[dl]&*=0{}\ar@{-}[r]\ar@{-}[dl]&*=0{}\ar@{-}[r]&*=0{}\ar@{-}[r]&*=0{}\ar@{-}[r]&*=0{}\ar@{-}[r]&*=0{}\ar@{-}[r]&*=0{}
\\*=0{}&*=0{}&*=0{}\ar@{-}[r]\ar@{-}[dl]&*=0{}\ar@{-}[r]\ar@{-}[dl]&*=0{}\ar@{-}[r]&*=0{}
\\*=0{}&*=0{}\ar@{-}[r]\ar@{-}[dl]&*=0{}\ar@{-}[dl]
\\*=0{}\ar@{-}[r]&*=0{}
\\
  *=0{}&*=0{}&*=0{}&*=0{}&*=0{}\ar@{-}[r]\ar@{-}[dl]&*=0{}\ar@{-}[r]\ar@{-}[dl]&*=0{}\ar@{-}[r]\ar@{-}[dl]&*=0{}\ar@{-}[r]\ar@{-}[dl]&*=0{}\ar@{-}[r]\ar@{-}[dl]&*=0{}\ar@{-}[dl]
\\*=0{}&*=0{}&*=0{}&*=0{}\ar@{-}[r]\ar@{-}[dl]&*=0{}\ar@{-}[r]\ar@{-}[dl]&*=0{}\ar@{-}[r]\ar@{-}[dl]&*=0{}\ar@{-}[r]&*=0{}\ar@{-}[r]&*=0{}
\\*=0{}&*=0{}&*=0{}\ar@{-}[r]\ar@{-}[dl]&*=0{}\ar@{-}[r]\ar@{-}[dl]&*=0{}\ar@{-}[dl]
\\*=0{}&*=0{}\ar@{-}[r]&*=0{}\ar@{-}[r]&*=0{}
\\
  *=0{}&*=0{}&*=0{}&*=0{}&*=0{}\ar@{-}[r]\ar@{-}[dl]&*=0{}\ar@{-}[r]\ar@{-}[dl]&*=0{}\ar@{-}[r]\ar@{-}[dl]&*=0{}\ar@{-}[r]\ar@{-}[dl]&*=0{}\ar@{-}[dl]
\\*=0{}&*=0{}&*=0{}&*=0{}\ar@{-}[r]&*=0{}\ar@{-}[r]&*=0{}\ar@{-}[r]&*=0{}\ar@{-}[r]&*=0{}
\\*=0{}}
$$
and
$$
A \otimes_5 B=\xymatrix @W=1.0pc @H=1.0pc @R=1.0pc @C=1.0pc
 {*=0{}&*=0{}&*=0{}&*=0{}&*=0{}&*=0{}&*=0{}&*=0{}\ar@{-}[r]\ar@{-}[dl]&*=0{}\ar@{-}[r]\ar@{-}[dl]&*=0{}\ar@{-}[r]\ar@{-}[dl]&*=0{}\ar@{-}[r]\ar@{-}[dl]&*=0{}\ar@{-}[dl]
\\*=0{}&*=0{}&*=0{}&*=0{}&*=0{}&*=0{}&*=0{}\ar@{-}[r]\ar@{-}[dl]&*=0{}\ar@{-}[r]\ar@{-}[dl]&*=0{}\ar@{-}[r]\ar@{-}[dl]&*=0{}\ar@{-}[r]&*=0{}
\\*=0{}&*=0{}&*=0{}&*=0{}&*=0{}&*=0{}\ar@{-}[r]\ar@{-}[dl]&*=0{}\ar@{-}[r]\ar@{-}[dl]&*=0{}\ar@{-}[dl]
\\*=0{}&*=0{}&*=0{}&*=0{}&*=0{}\ar@{-}[r]\ar@{-}[dl]&*=0{}\ar@{-}[r]\ar@{-}[dl]&*=0{}\ar@{-}[dl]
\\*=0{}&*=0{}&*=0{}&*=0{}\ar@{-}[r]\ar@{-}[dl]&*=0{}\ar@{-}[r]\ar@{-}[dl]&*=0{}
\\*=0{}&*=0{}&*=0{}\ar@{-}[r]\ar@{-}[dl]&*=0{}\ar@{-}[dl]
\\*=0{}&*=0{}\ar@{-}[r]\ar@{-}[dl]&*=0{}\ar@{-}[dl]
\\*=0{}\ar@{-}[r]&*=0{}
\\
  *=0{}&*=0{}&*=0{}&*=0{}&*=0{}&*=0{}&*=0{}&*=0{}\ar@{-}[r]\ar@{-}[dl]&*=0{}\ar@{-}[r]\ar@{-}[dl]&*=0{}\ar@{-}[r]\ar@{-}[dl]&*=0{}\ar@{-}[r]\ar@{-}[dl]&*=0{}\ar@{-}[r]\ar@{-}[dl]&*=0{}\ar@{-}[r]\ar@{-}[dl]&*=0{}\ar@{-}[dl]
\\*=0{}&*=0{}&*=0{}&*=0{}&*=0{}&*=0{}&*=0{}\ar@{-}[r]\ar@{-}[dl]&*=0{}\ar@{-}[r]\ar@{-}[dl]&*=0{}\ar@{-}[r]\ar@{-}[dl]&*=0{}\ar@{-}[r]\ar@{-}[dl]&*=0{}\ar@{-}[r]&*=0{}\ar@{-}[r]&*=0{}
\\*=0{}&*=0{}&*=0{}&*=0{}&*=0{}&*=0{}\ar@{-}[r]\ar@{-}[dl]&*=0{}\ar@{-}[r]\ar@{-}[dl]&*=0{}\ar@{-}[r]\ar@{-}[dl]&*=0{}
\\*=0{}&*=0{}&*=0{}&*=0{}&*=0{}\ar@{-}[r]\ar@{-}[dl]&*=0{}\ar@{-}[r]\ar@{-}[dl]&*=0{}
\\*=0{}&*=0{}&*=0{}&*=0{}\ar@{-}[r]\ar@{-}[dl]&*=0{}\ar@{-}[dl]
\\*=0{}&*=0{}&*=0{}\ar@{-}[r]\ar@{-}[dl]&*=0{}\ar@{-}[dl]
\\*=0{}&*=0{}\ar@{-}[r]&*=0{}
\\
  *=0{}&*=0{}&*=0{}&*=0{}&*=0{}&*=0{}&*=0{}&*=0{}\ar@{-}[r]\ar@{-}[dl]&*=0{}\ar@{-}[r]\ar@{-}[dl]&*=0{}\ar@{-}[r]\ar@{-}[dl]&*=0{}\ar@{-}[dl]
\\*=0{}&*=0{}&*=0{}&*=0{}&*=0{}&*=0{}&*=0{}\ar@{-}[r]\ar@{-}[dl]&*=0{}\ar@{-}[r]\ar@{-}[dl]&*=0{}\ar@{-}[r]\ar@{-}[dl]&*=0{}
\\*=0{}&*=0{}&*=0{}&*=0{}&*=0{}&*=0{}\ar@{-}[r]\ar@{-}[dl]&*=0{}\ar@{-}[r]\ar@{-}[dl]&*=0{}\ar@{-}[dl]
\\*=0{}&*=0{}&*=0{}&*=0{}&*=0{}\ar@{-}[r]\ar@{-}[dl]&*=0{}\ar@{-}[r]\ar@{-}[dl]&*=0{}\ar@{-}[dl]
\\*=0{}&*=0{}&*=0{}&*=0{}\ar@{-}[r]&*=0{}\ar@{-}[r]&*=0{}
\\
  *=0{}&*=0{}&*=0{}&*=0{}&*=0{}&*=0{}&*=0{}&*=0{}\ar@{-}[r]\ar@{-}[dl]&*=0{}\ar@{-}[r]\ar@{-}[dl]&*=0{}\ar@{-}[r]\ar@{-}[dl]&*=0{}\ar@{-}[r]\ar@{-}[dl]&*=0{}\ar@{-}[dl]
\\*=0{}&*=0{}&*=0{}&*=0{}&*=0{}&*=0{}&*=0{}\ar@{-}[r]&*=0{}\ar@{-}[r]&*=0{}\ar@{-}[r]&*=0{}\ar@{-}[r]&*=0{}
\\*=0{}}
$$
Here the alternative $\otimes_3^{'}$ = piecewise--piecewise max also turns oute to give $A\otimes_3^{'}B = A.$
\item
It might be nice to retain the geometric picture of the products of tableau shapes in terms of vertical
and horizontal addition, and addition in other directions as dimension increases. This is not found
in the current version of iteration.
This ``tableau stacking'' point of view is restored
if we restrict to sequences of tableau shapes that are decreasing in columns as well as rows.
These are lexicographically decreasing sequences of decreasing sequences so already obey the requirement that
${A_n}_1$ is decreasing in $n$. We expand this to require that ${A_n}_k$ be decreasing in $n$ for constant $k$,
which implies decreasing in $k$ for constant $n.$ We can represent these objects as infinite matrices with finitely nonzero
natural number entries, and with monotone decreasing columns and rows. We choose the sequence of rows to be the sequence
of sequences, i.e. each row represents a tableau which we draw as being parallel to the $xy$ plane. 
This choice is important because it determines the total ordering of matrices and thus the morphisms of the category.
Thus
horizontal composition is horizontal concatenation (diregarding zeroes) of matrices followed by sorting the new longer rows. Piecewise 
addition is addition of matrices. Now we define vertical ($z$-axis) addition as vertical concatentation of matrices followed
by sorting the new long columns. 

%Then we define a product for which $((A\otimes B)_n)_k = (/sort({A_i}_k,{B_i}_k)_{i=1}^{l(A)+l(B)})_n.$ 

Here is a visual example of the vertical addition, labeled $\otimes_1$: if
$$
A=\xymatrix @W=1.0pc @H=1.0pc @R=1.0pc @C=1.0pc
 {*=0{}&*=0{}&*=0{}&*=0{}&*=0{}\ar@{-}[r]\ar@{-}[dl]&*=0{}\ar@{-}[r]\ar@{-}[dl]&*=0{}\ar@{-}[r]\ar@{-}[dl]&*=0{}\ar@{-}[r]\ar@{-}[dl]&*=0{}\ar@{-}[dl]
\\*=0{}&*=0{}&*=0{}&*=0{}\ar@{-}[r]\ar@{-}[dl]&*=0{}\ar@{-}[r]\ar@{-}[dl]&*=0{}\ar@{-}[r]\ar@{-}[dl]&*=0{}\ar@{-}[r]&*=0{}
\\*=0{}&*=0{}&*=0{}\ar@{-}[r]\ar@{-}[dl]&*=0{}\ar@{-}[r]\ar@{-}[dl]&*=0{}\ar@{-}[dl]
\\*=0{}&*=0{}\ar@{-}[r]\ar@{-}[dl]&*=0{}\ar@{-}[r]\ar@{-}[dl]&*=0{}
\\*=0{}\ar@{-}[r]&*=0{}
\\
*=0{}&*=0{}&*=0{}&*=0{}&*=0{}\ar@{-}[r]\ar@{-}[dl]&*=0{}\ar@{-}[r]\ar@{-}[dl]&*=0{}\ar@{-}[r]\ar@{-}[dl]&*=0{}\ar@{-}[r]\ar@{-}[dl]&*=0{}\ar@{-}[dl]
\\*=0{}&*=0{}&*=0{}&*=0{}\ar@{-}[r]\ar@{-}[dl]&*=0{}\ar@{-}[r]\ar@{-}[dl]&*=0{}\ar@{-}[r]\ar@{-}[dl]&*=0{}\ar@{-}[r]&*=0{}
\\*=0{}&*=0{}&*=0{}\ar@{-}[r]\ar@{-}[dl]&*=0{}\ar@{-}[r]\ar@{-}[dl]&*=0{}
\\*=0{}&*=0{}\ar@{-}[r]\ar@{-}[dl]&*=0{}\ar@{-}[dl]
\\*=0{}\ar@{-}[r]&*=0{}
\\
*=0{}&*=0{}&*=0{}&*=0{}&*=0{}\ar@{-}[r]\ar@{-}[dl]&*=0{}\ar@{-}[r]\ar@{-}[dl]&*=0{}\ar@{-}[r]\ar@{-}[dl]&*=0{}\ar@{-}[dl]
\\*=0{}&*=0{}&*=0{}&*=0{}\ar@{-}[r]\ar@{-}[dl]&*=0{}\ar@{-}[r]\ar@{-}[dl]&*=0{}\ar@{-}[r]\ar@{-}[dl]&*=0{}
\\*=0{}&*=0{}&*=0{}\ar@{-}[r]\ar@{-}[dl]&*=0{}\ar@{-}[r]\ar@{-}[dl]&*=0{}
\\*=0{}&*=0{}\ar@{-}[r]&*=0{}
\\
*=0{}&*=0{}&*=0{}&*=0{}&*=0{}\ar@{-}[r]\ar@{-}[dl]&*=0{}\ar@{-}[r]\ar@{-}[dl]&*=0{}\ar@{-}[r]\ar@{-}[dl]&*=0{}\ar@{-}[dl]&*=0{}
\\*=0{}&*=0{}&*=0{}&*=0{}\ar@{-}[r]&*=0{}\ar@{-}[r]&*=0{}\ar@{-}[r]&*=0{}
\\*=0{}}
\text{ and }B=\xymatrix @W=1.0pc @H=1.0pc @R=1.0pc @C=1.0pc
 {*=0{}&*=0{}&*=0{}&*=0{}\ar@{-}[r]\ar@{-}[dl]&*=0{}\ar@{-}[r]\ar@{-}[dl]&*=0{}\ar@{-}[dl]
\\*=0{}&*=0{}&*=0{}\ar@{-}[r]\ar@{-}[dl]&*=0{}\ar@{-}[r]\ar@{-}[dl]&*=0{}
\\*=0{}&*=0{}\ar@{-}[r]\ar@{-}[dl]&*=0{}\ar@{-}[dl]
\\*=0{}\ar@{-}[r]&*=0{}
\\
  *=0{}&*=0{}&*=0{}&*=0{}\ar@{-}[r]\ar@{-}[dl]&*=0{}\ar@{-}[r]\ar@{-}[dl]&*=0{}\ar@{-}[dl]&*=0{}
\\*=0{}&*=0{}&*=0{}\ar@{-}[r]\ar@{-}[dl]&*=0{}\ar@{-}[r]\ar@{-}[dl]&*=0{}
\\*=0{}&*=0{}\ar@{-}[r]&*=0{}
\\
  *=0{}&*=0{}&*=0{}&*=0{}\ar@{-}[r]\ar@{-}[dl]&*=0{}\ar@{-}[r]\ar@{-}[dl]&*=0{}\ar@{-}[dl]
\\*=0{}&*=0{}&*=0{}\ar@{-}[r]&*=0{}\ar@{-}[r]&*=0{}
\\*=0{}}
$$
then we let
$$
A \otimes_1 B=\xymatrix @W=1.0pc @H=1.0pc @R=1.0pc @C=1.0pc
 {*=0{}&*=0{}&*=0{}&*=0{}&*=0{}\ar@{-}[r]\ar@{-}[dl]&*=0{}\ar@{-}[r]\ar@{-}[dl]&*=0{}\ar@{-}[r]\ar@{-}[dl]&*=0{}\ar@{-}[r]\ar@{-}[dl]&*=0{}\ar@{-}[dl]
 \\*=0{}&*=0{}&*=0{}&*=0{}\ar@{-}[r]\ar@{-}[dl]&*=0{}\ar@{-}[r]\ar@{-}[dl]&*=0{}\ar@{-}[r]\ar@{-}[dl]&*=0{}\ar@{-}[r]&*=0{}
 \\*=0{}&*=0{}&*=0{}\ar@{-}[r]\ar@{-}[dl]&*=0{}\ar@{-}[r]\ar@{-}[dl]&*=0{}\ar@{-}[dl]
 \\*=0{}&*=0{}\ar@{-}[r]\ar@{-}[dl]&*=0{}\ar@{-}[r]\ar@{-}[dl]&*=0{}
 \\*=0{}\ar@{-}[r]&*=0{}
 \\
 *=0{}&*=0{}&*=0{}&*=0{}&*=0{}\ar@{-}[r]\ar@{-}[dl]&*=0{}\ar@{-}[r]\ar@{-}[dl]&*=0{}\ar@{-}[r]\ar@{-}[dl]&*=0{}\ar@{-}[r]\ar@{-}[dl]&*=0{}\ar@{-}[dl]
 \\*=0{}&*=0{}&*=0{}&*=0{}\ar@{-}[r]\ar@{-}[dl]&*=0{}\ar@{-}[r]\ar@{-}[dl]&*=0{}\ar@{-}[r]\ar@{-}[dl]&*=0{}\ar@{-}[r]&*=0{}
 \\*=0{}&*=0{}&*=0{}\ar@{-}[r]\ar@{-}[dl]&*=0{}\ar@{-}[r]\ar@{-}[dl]&*=0{}
 \\*=0{}&*=0{}\ar@{-}[r]\ar@{-}[dl]&*=0{}\ar@{-}[dl]
 \\*=0{}\ar@{-}[r]&*=0{}
 \\
 *=0{}&*=0{}&*=0{}&*=0{}&*=0{}\ar@{-}[r]\ar@{-}[dl]&*=0{}\ar@{-}[r]\ar@{-}[dl]&*=0{}\ar@{-}[r]\ar@{-}[dl]&*=0{}\ar@{-}[dl]
 \\*=0{}&*=0{}&*=0{}&*=0{}\ar@{-}[r]\ar@{-}[dl]&*=0{}\ar@{-}[r]\ar@{-}[dl]&*=0{}\ar@{-}[r]\ar@{-}[dl]&*=0{}
 \\*=0{}&*=0{}&*=0{}\ar@{-}[r]\ar@{-}[dl]&*=0{}\ar@{-}[r]\ar@{-}[dl]&*=0{}
 \\*=0{}&*=0{}\ar@{-}[r]&*=0{}
\\
  *=0{}&*=0{}&*=0{}&*=0{}&*=0{}\ar@{-}[r]\ar@{-}[dl]&*=0{}\ar@{-}[r]\ar@{-}[dl]&*=0{}\ar@{-}[r]\ar@{-}[dl]&*=0{}\ar@{-}[dl]&*=0{}
  \\*=0{}&*=0{}&*=0{}&*=0{}\ar@{-}[r]\ar@{-}[dl]&*=0{}\ar@{-}[r]\ar@{-}[dl]&*=0{}\ar@{-}[r]&*=0{}&*=0{}
  \\*=0{}&*=0{}&*=0{}\ar@{-}[r]\ar@{-}[dl]&*=0{}\ar@{-}[dl]
  \\*=0{}&*=0{}\ar@{-}[r]&*=0{}
  \\
    *=0{}&*=0{}&*=0{}&*=0{}&*=0{}\ar@{-}[r]\ar@{-}[dl]&*=0{}\ar@{-}[r]\ar@{-}[dl]&*=0{}\ar@{-}[dl]&*=0{}
  \\*=0{}&*=0{}&*=0{}&*=0{}\ar@{-}[r]\ar@{-}[dl]&*=0{}\ar@{-}[r]\ar@{-}[dl]&*=0{}
  \\*=0{}&*=0{}&*=0{}\ar@{-}[r]&*=0{}
\\
  *=0{}&*=0{}&*=0{}&*=0{}&*=0{}\ar@{-}[r]\ar@{-}[dl]&*=0{}\ar@{-}[r]\ar@{-}[dl]&*=0{}\ar@{-}[dl]
  \\*=0{}&*=0{}&*=0{}&*=0{}\ar@{-}[r]&*=0{}\ar@{-}[r]&*=0{}&*=0{}
\\
  *=0{}&*=0{}&*=0{}&*=0{}&*=0{}\ar@{-}[r]\ar@{-}[dl]&*=0{}\ar@{-}[r]\ar@{-}[dl]&*=0{}\ar@{-}[dl]
  \\*=0{}&*=0{}&*=0{}&*=0{}\ar@{-}[r]&*=0{}\ar@{-}[r]&*=0{}
\\*=0{}}
$$
Note that in this restricted setting of decreasing matrices the lexicographic sorting of sequences (rows) of 
two operands in a product does not preserve the decreasing property. Thus we consider whether
the new product is incorporated via interchanges 
into the 5-fold monoidal category described above as containing the entrywise
max as a product of matrices.

Then note that the natural place for this new product is in a orthogonal category to the previously described
5-fold monoidal
category of matrices (sequence of rows) that has the same objects (but seen as sequences of columns.) The ordering
is lex order on the columns, and there is also a piecewise lex max on the columns and a piecewise sorting of the columns.
Integrating the two orthogaonal categories is nontrivial.
A nonexample is easy to realize when we try to find an interchange between for instance the 
overall max on matrices ordered lexicographically by rows and the overall max when ordered by columns

It is true however that at least the three products that preserve the total sum of the entries in both
matrices do interact via interchanges. Renumbered they are: 
$\otimes_1$ is the vertical concatentation of matrices followed
by sorting the new longer columns, $\otimes_2$ is
horizontal concatenation of matrices followed by sorting the new longer rows and $\otimes_3$ is the
addition of matrices. We already have existence of $\eta^{13}$ and $\eta^{23}$ by previous arguments,
we need only check for $\eta^{12}.$ This existence is clearly the case since we are ordering the matrices
by giving precedence to the rows. Thus sorting columns first and then rows is guaranteed to
give something larger lexicographically than sorting horizontally first.

 
\end{enumerate}


   
\newpage   
\MySection{$n$-fold operads}    
      Let ${\cal V}$ be an $n$-fold monoidal category as defined in Section 2.
      \begin{definition}
      For $2\le m\le n$ an $m$-fold operad ${\cal C}$ in ${\cal V}$ consists of objects ${\cal C}(j)$, $j\ge 1$, 
      a unit map ${\cal J}:I\to {\cal C}(1)$, 
      %a right action by the symmetric group
      %$\Sigma_j$ on ${\cal C}(j)$ for each $j$ 
      and composition maps in ${\cal V}$
      $$
      \gamma^{pq}:{\cal C}(k) \otimes_p ({\cal C}(j_1) \otimes_q \dots \otimes_q {\cal C}(j_k))\to {\cal C}(j)
      $$
      for $m\ge q>p \ge 1$, $k\ge 1$, $j_s\ge0$ for $s=1\dots k$ and $\sum\limits_{n=1}^k j_n = j$. The composition maps obey the following axioms
      \begin{enumerate}
      \item Associativity: The following diagram is required to commute for all $m\ge q>p \ge 1$, $k\ge 1$, $j_s\ge 0$ and $i_t\ge 0$, and
      where $\sum\limits_{s=1}^k j_s = j$ and $\sum\limits_{t=1}^j i_t = i.$ Let $g_s= \sum\limits_{u=1}^s j_u$ and
      let $h_s=\sum\limits_{u=1+g_{s-1}}^{g_s} i_u$.

      $$
      \xymatrix{
      {\cal C}(k)\otimes_p\left(\bigotimes\limits_{s=1}^k{}_q {\cal C}(j_s)\right)\otimes_p
                       \left(\bigotimes\limits_{t=1}^j{}_q {\cal C}(i_t)\right)
      \ar[rr]^>>>>>>>>>>>>{\gamma^{pq} \otimes_p \text{id}}
      \ar[dd]_{\text{id} \otimes_p \eta^{pq}}
      &&{\cal C}(j)\otimes_p \left(\bigotimes\limits_{t=1}^j{}_q {\cal C}(i_t)\right)
      \ar[d]^{\gamma^{pq}}
      \\
      &&{\cal C}(i)
      \\
      {\cal C}(k)\otimes_p \left(\bigotimes\limits_{s=1}^k{}_q {\cal C}(j_s)\otimes_p
                      \left(\bigotimes\limits_{u=1}^{j_s}{}_q {\cal C}(i_{u+g_{s-1}})\right)\right)
      \ar[rr]_>>>>>>>>>>{\text{id} \otimes_p(\otimes^k_q\gamma^{pq})}
      &&{\cal C}(k)\otimes_p\left(\bigotimes\limits_{s=1}^k{}_q {\cal C}(h_s)\right)
      \ar[u]_{\gamma^{pq}}
      }
      $$
      
      where the $\eta^{pq}$ on the left actually stands for a variety of equivalent maps which factor into instances of the
      $pq$ interchange. 
     
     Respect of units is required just as in the symmetric case.
          The following unit diagrams commute for all $m\ge q>p \ge 1$.
      $$
      \xymatrix{
      {\cal C}(k)\otimes_p (\otimes_q^k I)
      \ar[d]_{1\otimes_p(\otimes_q^k {\cal J})}
      \ar@{=}[r]^{}
      &{\cal C}(k)\\
      {\cal C}(k)\otimes_p(\otimes_q^k {\cal C}(1))
      \ar[ur]_{\gamma^{pq}}
      }
      \xymatrix{
      I\otimes_p {\cal C}(k)
      \ar[d]_{{\cal J}\otimes_p 1}
      \ar@{=}[r]^{}
      &{\cal C}(k)\\
      {\cal C}(1)\otimes_p {\cal C}(k)
      \ar[ur]^{\gamma^{pq}}
      }
      $$
      \end{enumerate}
      \end{definition}
      
      \begin{example}
      ~
      \end{example}
      One large family of operads in ModSeq({\bf N}) is that of natural number indexed collections of tableau shapes
       ${\cal C}(n), n\in${\bf N}, such that
      $h({\cal C}(n)) = f(n)$ where $f:${\bf N}$\to${\bf N} is a function such 
      that $f(1) = 0$ and $f(i+j)\ge f(i) + f(j).$
      These conditions are not necessary, but they are sufficient since the first implies that ${\cal C}(1) =0$ which shows that the
      unit conditions are satisfied; and the second implies that the maps $\gamma$ exist.
      We see existence of $\gamma^{12}$ since 
      $h({\cal C}(k)\otimes_1({\cal C}(j_1)\otimes_2...\otimes_2{\cal C}(j_k))) =\max(f(k),\max(f(j_i))) \le f(j)  =h({\cal C}(j)).$
      We also have existence of $\gamma^{13}$ and $\gamma^{23}$ since $\max(f(k),\max(f(j_i))) \le f(j).$
      We have existence of $\gamma^{14}$ ,$\gamma^{24}$ and $\gamma^{34}$ since
      $ \max(f(k), \sum{f(j_i)}) \le f(j)$.  
      

      
      Examples of $f$ include  $(x-1)P(x)$ where $P$ is a polynomial with coefficients in {\bf N}. This is easy to show
      since then $P$ will be monotone increasing for 
      $x\ge 1$ and thus $(i+j-1)P(i+j) = (i-1)P(i+j)+jP(i+j) \ge (i-1)P(i)+jP(j) - P(j).$
      By this argument we can also use any $f= (x-1)g(x)$ where $g:${\bf N}$\to${\bf N} is monotone  increasing 
      for $x \ge 1.$
      
      For a specific example with a 
      handy picture that also illustrates again the nontrivial use of the interchange $\eta$
      we simply let $f = x-1.$ Then we have to actually describe the elements of ModSeq({\bf N}) that make up the
      operad. One nice choice is the operad ${\cal C}$ where ${\cal C}(n) = \{n-1,n-1,...,n-1\},$ the  
      $(n-1)\times (n-1)$ square tableau shape. 
      
      ${\cal C}(1) = 0$, ${\cal C}(2) = \xymatrix@W=1.2pc @H=1.2pc @R=0pc @C=0pc @*[F-]{~}$, 
      ${\cal C}(3) = \xymatrix@W=1.2pc @H=1.2pc @R=0pc @C=0pc @*[F-]{~&~\\~&~} \dots$
      
      For instance $\gamma^{34}:{\cal C}(3)\otimes_3 ({\cal C}(1)\otimes_4{\cal C}(3)\otimes_4{\cal C}(2)) \to {\cal C}(6)$
      appears as the relation
      $$
      \xymatrix@W=1.2pc @H=1.2pc @R=0pc @C=0pc @*[F-]{~&~&~&~\\~&~&~&~\\~} \le \xymatrix@W=1.2pc @H=1.2pc @R=0pc @C=0pc @*[F-]{~&~&~&~&~\\~&~&~&~&~\\~&~&~&~&~\\~&~&~&~&~\\~&~&~&~&~}
      $$
      An instance of the associativity diagram with upper left position 
      ${\cal C}(2)\otimes_3({\cal C}(3)\otimes_4{\cal C}(2))\otimes_3
      ({\cal C}(2)\otimes_4{\cal C}(2)\otimes_4{\cal C}(4)\otimes_4{\cal C}(5)\otimes_4{\cal C}(3))$ is as follows:
      
      \begin{tabular}{llll}
      &$\xymatrix@W=.7pc @H=.7pc @R=0pc @C=0pc @*[F-]{~&~&~&~&~&~&~\\~&~&~&~&~&~\\~&~&~&~&~\\~&~&~&~\\~&~&~\\~&~&~\\~&~&~\\~&~\\~&~\\~\\~}$&$\to$&$\xymatrix@W=.7pc @H=.7pc @R=0pc @C=0pc @*[F-]{~&~&~&~&~&~&~&~\\~&~&~&~&~&~&~&~\\~&~&~&~&~&~&~&~\\~&~&~&~&~&~&~&~\\~&~&~\\~&~&~\\~&~&~\\~&~\\~&~\\~\\~}$ \\
      &&&$\downarrow$\\
      &$\downarrow$&&$\xymatrix@W=1.7pc @H=1.7pc @R=0pc @C=0pc @*[F-]{15\times 15 \text{ square }}$\\
      &&&$\uparrow$\\
      &$\xymatrix@W=.7pc @H=.7pc @R=0pc @C=0pc @*[F-]{~&~&~&~&~&~\\~&~&~&~&~\\~&~&~&~&~\\~&~&~&~\\~&~&~&~\\~&~&~&~\\~&~&~\\~&~\\~&~\\~\\~}$&$\to$&$\xymatrix@W=.7pc @H=.7pc @R=0pc @C=0pc @*[F-]{~&~&~&~&~&~&~&~\\~&~&~&~&~&~&~\\~&~&~&~&~&~&~\\~&~&~&~&~&~&~\\~&~&~&~&~&~&~\\~&~&~&~&~&~&~\\~&~&~&~&~&~&~\\~&~&~&~&~&~&~\\~&~&~&~&~&~&~\\~&~&~&~&~&~&~\\~&~&~&~&~&~&~\\~&~&~&~&~&~&~\\~&~&~&~&~&~&~\\~&~&~&~&~&~&~}$\\
      \end{tabular}

      
We conclude with a description of the concepts of $n$-fold operad algebra and of the tensor products of operads
and algebras.
\begin{definition}\label{opalg}
Let ${\cal C}$ be  an $n$-fold operad in ${\cal V}$. A  ${\cal C}$-algebra is an object $A\in{\cal V}$ and maps
$$
\theta^{pq}:{\cal C}(j)\otimes_p(\otimes_q^j A)\to A
$$
for $n\ge q>p \ge 1$, $j\ge 0$. 
\begin{enumerate}
      \item Associativity: The following diagram is required to commute for all $n\ge q>p \ge 1$, 
      $k\ge 1$, $j_s\ge 0$ , and
      where $\sum\limits_{s=1}^k j_s = j.$
$$
\xymatrix{{\cal C}(k) \otimes_p ({\cal C}(j_1) \otimes_q \dots \otimes_q {\cal C}(j_k))\otimes_p(\otimes_q^j A)
\ar[rr]^>>>>>>>>>>>>{\gamma^{pq} \otimes_p \text{id}}
      \ar[dd]_{\text{id} \otimes_p \eta^{pq}}
  && {\cal C}(j)\otimes_p(\otimes_q^j A)
  \ar[d]_{\theta^{pq}}\\
  &&A\\
  {\cal C}(k) \otimes_p (({\cal C}(j_1)\otimes_p (\otimes_q^{j_1} A))\otimes_q \dots \otimes_q ({\cal C}(j_k)\otimes_p (\otimes_q^{j_k} A)))
  \ar[rr]^>>>>>>>>>>>>{\text{id} \otimes_p (\otimes_q^k \theta^{pq})}
  &&{\cal C}(k)\otimes_p(\otimes_q^k A)
  \ar[u]^{\theta^{pq}}
}
$$
\item Units: The following diagram is required to commute for all $n\ge q>p \ge 1$.
$$
\xymatrix{
      I\otimes_p A
      \ar[d]_{{\cal J}\otimes_p 1}
      \ar@{=}[r]^{}
      &A\\
      {\cal C}(1)\otimes_p A
      \ar[ur]^{\theta^{pq}}
      }
$$
\end{enumerate}
\end{definition}

\begin{definition}\label{tensor}
      Let ${\cal C}$,${\cal D}$ be  $n$-fold operads.
For $1 \le i \le (n-2)$ and using a $\otimes'_k$ to denote the product of
two $n$-fold operads, we define that product to be given by:

$$({\cal C}\otimes'_i {\cal D})(j) = {\cal C}(j) \otimes_{i+2} {\cal D}(j).$$

\end{definition}

That the product of $n$-fold operads is itself an $n$-fold operad is easy to verify once 
we note that the new $\gamma$ is in terms of the two old ones:
$$
\gamma^{pq}_{{\cal C}\otimes'_i {\cal D}} = 
(\gamma^{pq}_{{\cal C}}\otimes_{i+2}\gamma^{pq}_{{\cal D}})\circ \eta^{p(i+2)} \circ (1 \otimes_p \eta^{q(i+2)})
$$

where the subscripts denote the
$n$-fold operad the $\gamma$ belongs to and the $\eta$'s actually stand for any of 
the equivalent maps which factor into them. 


      
     If $A$ is an algebra of ${\cal C}$ and $B$ is an algebra of ${\cal D}$ then
           $A\otimes_{i+2} B$ is an algebra for ${\cal C}\otimes'_i{\cal D}.$
     
 That the product of $n$-fold operad algebras is itself an $n$-fold operad algebra is easy to verify once 
we note that the new $\theta$ is in terms of the two old ones: 
  $$
     \theta^{pq}_{A\otimes_{i+2} B} = 
     (\theta^{pq}_{A}\otimes_{i+2}\theta^{pq}_{B})\circ \eta^{p(i+2)} \circ (1 \otimes_p \eta^{q(i+2)})
$$
    
Maps of operads and operad algebras are straightforward to describe--they are required to 
preserve all the structure in sight; that is to commute with $\gamma$ and ${\cal J}$ and respectively 
with $\theta.$
It is left as an exercise for the reader to verify the assertion that the $n$-fold
operads of a given category form themselves into an $(n-2)$-category, with interchange laws obeying 
all the axioms of Section 2 above. Basically the result follows almost immediately from
the coherence of iterated monoidal categories.
 
      
   %{bib2}
      
      
      
      
      
      
       
        % {bib}
        
        %\clearpage
        \newpage
        \refs
        \bibitem[Baez and Dolan, 1998]{Baez1}{J. C. Baez and J. Dolan, Categorification, in ``Higher Category Theory'', 
        eds. E. Getzler and M. Kapranov, Contemp. Math. 230 , American Mathematical Society, 1-36, (1998).}
        \bibitem[Balteanu et.al, 2003]{Balt}{C. Balteanu, Z. Fiedorowicz, R. Schw${\rm \ddot a}$nzl, R. Vogt, 
        Iterated Monoidal Categories,
        Adv. Math. 176 (2003), 277-349.}
        \bibitem[Batanin,1998]{bat}{M. Batanin, Monoidal globular categories as a natural environment
 for the theory of weak n-categories, Advances in Math 136, (1998) 39-103.}
        \bibitem[Borceux, 1994]{Borc}{F. Borceux, Handbook of Categorical Algebra 1: 
        Basic Category Theory, Cambridge University Press, 1994.}
        \bibitem[Boardman and Vogt, 1973]
            {BV1}{J. M. Boardman and  R. M. Vogt, Homotopy invariant algebraic 
            structures on topological spaces, Lecture Notes in Mathematics, Vol. 347, 
            Springer, 1973.}
        %\bibitem[Day, 1970]{Day}{B.J. Day, On closed categories of functors, Lecture Notes in 
        %Math 137 (Springer, 1970) 1-38}
        %\bibitem[Eilenberg and Kelly, 1965]{EK1}{S. Eilenberg and G. M. Kelly, Closed Categories, 
        %Proc. Conf. on Categorical Algebra, 
        %Springer-Verlag (1965), 421-562. }
        \bibitem[Fiedorowicz]
	       {ZF}{Z. Fiedorowicz, The symmetric bar construction, preprint.}
	\bibitem[Forcey, 2004]{forcey1}{S. Forcey, Enrichment Over Iterated Monoidal Categories,
        Algebraic and Geometric Topology 4 (2004), 95-119.}
        \bibitem[Forcey2, 2004]{forcey2}{S. Forcey, Vertically iterated classical enrichment, 
            Theory and Applications of Categories,  12 (2004) No. 10, 299-325.}   
       %\bibitem[Forcey, 2003]
       %{forcey2}{S. Forcey, Higher Dimensional Enrichment, preprint math.CT/0306086, 2003.}
        \bibitem[Joyal and Street, 1993]{JS}{A. Joyal and R. Street, 
        Braided tensor categories, Advances in Math. 102(1993), 20-78.}
         \bibitem[Kelly, 1982]{Kelly}{G. M. Kelly, Basic Concepts of Enriched Category Theory, London Math. 
        Society Lecture Note Series 64, 1982.}
        \bibitem[Leinster, 2004]{lst}{T. Leinster, Operads in higher-dimensional category theory,
        Theory and Applications of Categories 12 (2004) No. 3, 73-194.}
        \bibitem[Litvinov and Sobolevskii, 2001]{LitSob}{G. Litvinov and A. Sobolevskii, Idempotent 
        Interval Analysis and Optimization Problems, Reliable Computing, vol. 7, no. 5 (2001), 353-377.}
       % \bibitem[Lyubashenko, 2003]{Lyub}{V. Lyubashenko, Category of $A_{\infty}$-categories, 
       % Homology, Homotopy and Applications 5(1) (2003), 1-48.}
       \bibitem[Mac Lane, 1965]
	       {Mac}
        {S. MacLane, Categorical algebra, Bull. A. M. S. 71(1965), 40-106.}
        \bibitem[Mac Lane, 1998]{MacLane}{S. Mac Lane, Categories for the 
        Working Mathematician 2nd. edition, Grad. Texts in Math. 5, 1998.}
         \bibitem[May, 1972]
	     {May}{J. P. May, The geometry of iterated loop spaces, Lecture Notes in
              Mathematics, Vol. 271, Springer, 1972}
         \bibitem[May, 2001]{may2}{J.P. May, Operadic categories, $A_{\infty}$-categories, and $n$-categories, Notes
         of a talk given at Morelia, Mexico on May 25 2001.}     
        \bibitem[Schr\"{o}der, 2003]{Schrod}{Bernd S.W. Schr\"{o}der, Ordered Sets, an Introduction, Birkh\"{a}user, 2003.} 
        \bibitem[Speyer and Sturmfels, 2004]{Sturm}{D. Speyer and B. Sturmfels, Tropical Mathematics, preprint math.CO/0408099, 2004.}
        \bibitem[Stasheff, 1963]
             {Sta}{J. D. Stasheff, Homotopy associativity of H-spaces I, Trans. A. M.         
             S. 108(1963), 275-292.}
        \bibitem[Trimble, 1999]{trimble}{What are fundamental $n$-groupoids? Seminar at DPMMS, Cambridge, 24 August 1999.}     
        \bibitem[Weber, 2004]{web}{M. Weber, Operads within monoidal pseudo algebras, preprint available at
        www.mathstat.uottawa.ca/~mwebe937/research/Preprints.html, 2004.} 
         \endrefs
    
\end{document}
