%%%%%%%%%%%%%%%%%%%%%%%%%%%%%%%%%%%%%%%%%%%%%%%%%%%%%%%%%%%%%%%%%%%%%%%%%%%%%%%%%%%%%
%
%Last Edited 07/14/08 by SD
%
%%%%%%%%%%%%%%%%%%%%%%%%%%%%%%%%%%%%%%%%%%%%%%%%%%%%%%%%%%%%%%%%%%%%%%%%%%%%%%%%%%%%%

\documentclass[10pt]{amsart}
\usepackage{amsmath,amsthm,amssymb}
\usepackage{graphicx}
\usepackage[margin=1in]{geometry}
%\usepackage{fullpage}
\usepackage[all]{xy}
\xyoption{v2} \xyoption{knot}

\let\url=\undefined
\usepackage
[bookmarks,colorlinks,linkcolor=blue,citecolor=blue,urlcolor=blue] {hyperref}



%%%%%%%%%%%%%%%%%%%%%%%%%%%%%%%%%%%%%%%%%%%%%%%%%%%%%%%%%%%%%%%%%%%%%%%%%%%%%%%%%%%%%
%  Group Notation
%%%%%%%%%%%%%%%%%%%%%%%%%%%%%%%%%%%%%%%%%%%%%%%%%%%%%%%%%%%%%%%%%%%%%%%%%%%%%%%%%%%%%

\newcommand{\PGL} {\Pj\Gl_2(\R)}                           %PGL_2(R)
\newcommand{\PGLC} {\Pj\Gl_2(\C)}                          %PGL_2(C)
\newcommand{\RP} {\R\Pj^1}                                 %RP^1
\newcommand{\CP} {\C\Pj^1}                                 %CP^1
\newcommand{\C} {{\mathbb C}}                              %complex
\newcommand{\R} {{\mathbb R}}                              %reals
\newcommand{\Z} {{\mathbb Z}}                              %integers
\newcommand{\Pj} {{\mathbb P}}                             %projective space
\newcommand{\Sg} {{\mathbb S}}                             %symmetric group
\newcommand{\Gl} {{\rm Gl}}                                %g.linear group

\newcommand{\suchthat} {\:\: | \:\:}
\newcommand{\ore} {\ \ {\it or} \ \ }
\newcommand{\oand} {\ \ {\it and} \ \ }

\newcommand{\oM} [1] {\ensuremath{{\mathcal M}_{0,#1}(\R)}}                 %open
\newcommand{\M} [1] {\ensuremath{{\overline{\mathcal M}}{_{0, #1}(\R)}}}    %compact DMK
\newcommand{\cM} [1] {\ensuremath{{\mathcal M}_{0,#1}}}                     %open
\newcommand{\CM} [1] {\ensuremath{{\overline{\mathcal M}}{_{0, #1}}}}       %compact DMK

\newcommand{\Tubeset} {\mathfrak{T}}                                        %set of all tubings

\newcommand{\D} {\Delta}
\newcommand{\K} {\mathcal{K}}                           %Polytope
\newcommand{\J} {\mathcal{J}}

\newcommand{\sm}{\varepsilon}

\newcommand{\rec}[1] {G^*(#1)}                  %reconnected complement

\newcommand{\KG} {{\K} G}
\newcommand{\JG} {{\J} G}                %graph multiplihedra

\newcommand{\temp} {\nabla}
%\newcommand{\temp} {\underline{\mathbf{\Delta}}}
\newcommand{\JGr} {\JG_r}
\newcommand{\JGd} {\JG_d}


\newcommand{\tubeA}{\raisebox{-.5ex}{\scalebox{.27}{\includegraphics{signat.pdf}}}}
\newcommand{\tubeB}{\raisebox{-.6ex}{\scalebox{.2}{\includegraphics{ribbon.pdf}}}}
\newcommand{\tubeC}{\raisebox{-.6ex}{\includegraphics{tube-C.eps}}}

\def\cal#1{\mathcal{#1}}
\newcommand{\bcal}[1]{\mbox{\boldmath${\cal {#1}}$}}

%%%%%%%%%%%%%%%%%%%%%%%%%%%%%%%%%%%%%%%%%%%%%%%%%%%%%%%%%%%%%%%%%%%%%%%%%%%%%%%%%%%%%%%%%%%
%
%  paper formatting
%
%%%%%%%%%%%%%%%%%%%%%%%%%%%%%%%%%%%%%%%%%%%%%%%%%%%%%%%%%%%%%%%%%%%%%%%%%%%%%%%%%%%%%%%%%%%

\theoremstyle{plain}
\newtheorem{thm}{Theorem}
\newtheorem{prop}[thm]{Proposition}
\newtheorem{cor}[thm]{Corollary}
\newtheorem{lem}[thm]{Lemma}
\newtheorem{conj}[thm]{Conjecture}

\theoremstyle{definition}
\newtheorem{defn}[thm]{Definition}
\newtheorem*{exmp}{Example}

\theoremstyle{remark}
\newtheorem*{rem}{Remark}
\newtheorem*{hnote}{Historical Note}
\newtheorem*{nota}{Notation}
\newtheorem*{ack}{Acknowledgments}
\numberwithin{equation}{section}





\keywords{multiplihedron, graph associahedron, realization, convex hull}

%\maketitle

%%%%%%%%%%%%%%%%%%%%%%%%%%%%%%%%%%%%%%%%%%%%%%%%%%%%%%%%%%%%%%%%%%%%%%%%%%%%%%%%%%%%%
%NOTE: The \baselineskip=15pt below is the minimum needed to make the paper look decent.
%      There are many superscripts {\tilde, \overline} which cause erratic
%      spaces between sentences. \baselineskip corrects these problems.  Also see the
%      \baselineskip=12pt by References.
%%%%%%%%%%%%%%%%%%%%%%%%%%%%%%%%%%%%%%%%%%%%%%%%%%%%%%%%%%%%%%%%%%%%%%%%%%%%%%%%%%%%%

\baselineskip=14pt


\begin{document}
%\pagestyle{myheadings} \markboth{}{} \noindent \thispagestyle{empty}
%\vspace{0.5cm}
\noindent October 24, 2008\text{ }\text{ }\text{ }\text{ }\text{ }\text{ }\text{ }\text{ }\text{
}\text{ }\text{ }\text{ }\text{ }\text{ }\text{ }\text{ }\text{ }\ \tubeB
\begin{tabular}{llllll}\\
&&&&&Department of Physics and Mathematics  \\
&&&&&Tennessee State University \\
&&&&&3500 John A. Merritt Blvd. \\
&&&&&Nashville, TN 37209-1561, USA\\
&&&&\\
\end{tabular}
\noindent Chair, Department of Mathematics\\
 Syracuse University\\
 Syracuse, NY 13244\\
\\
Dear Dr. Poletsky and search committee,
\\\\
This letter accompanies my application for your open tenured or tenure track positions in
mathematics. I believe that my interests will fit well into the research community in your
department. Broadly, my research lies in the areas of combinatorial algebra, geometric
combinatorics, structured categories and homotopy theory. Projects with collaborators currently
underway include a study of the Hopf algebras and modules that arise from both geometric and
combinatorial objects such as the associahedron and multiplihedron. Another project involves the
classification and properties of combinatorial operads that live in structured categories. I am
interested in the interplay between low dimensional topology and higher dimensional categories, and
excited about the potential for applying all these abstract theories to questions in physics and
computer science.
\\\\
I am currently serving at the level of Assistant Professor at Tennessee State University. My
experience here has been wonderful, both as regards research opportunities and teaching
responsibilities. One of my primary volunteer activities has been to organize, run and fund (via
grant) our Research Seminar. A related project of mine is the development of our course entitled
``Research Experience.'' I have also advised more than a few students on their masters theses and
senior projects. All together, these experiences have reinforced my commitment to integrating
teaching and research -- a commitment that certainly would not be out of place in the Syracuse
mathematics department.
\\\\
Letters of reference will be provided by the following:
\\\\
 \begin{tabular} {llll}
 1. & Dr. Ronald Brown & Bangor University & ronnie.profbrown@btinternet.com \\
 2. & Dr. Satyan Devadoss & Williams College & Satyan.Devadoss@williams.edu\\
 3. & Dr. Sandra Scheick & Tennessee State University & sscheick@tnstate.edu (teaching) \\
 4.& Dr. Frank Sottile & Texas A\&M & sottile@math.tamu.edu \\
5. & Dr. James Stasheff & University of Pennsylvania & jds@math.upenn.edu\\
\end{tabular}
\\\\
Thank you for your time and consideration and please let me know if there is any more information
you need to help evaluate my application.
\\\\

\begin{tabular}{lllllll}
&&&Sincerely, \\
&&&\ \tubeA \ \ \\
&&&Dr. Stefan Forcey \\
&&&6521 Mercomatic Ct.\\
&&&Nashville, TN 37209 \\
&&&(615)963-5849\\
&&& sforcey@tnstate.edu\\
&&& \url{http://faculty.tnstate.edu/sforcey }\\
\end{tabular}
 \end{document}
